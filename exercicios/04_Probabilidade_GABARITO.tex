\documentclass[a4paper,11pt,fleqn]{article}\usepackage[]{graphicx}\usepackage[]{color}
%% maxwidth is the original width if it is less than linewidth
%% otherwise use linewidth (to make sure the graphics do not exceed the margin)
\makeatletter
\def\maxwidth{ %
  \ifdim\Gin@nat@width>\linewidth
    \linewidth
  \else
    \Gin@nat@width
  \fi
}
\makeatother

\definecolor{fgcolor}{rgb}{0, 0, 0}
\newcommand{\hlnum}[1]{\textcolor[rgb]{0,0,0}{#1}}%
\newcommand{\hlstr}[1]{\textcolor[rgb]{0,0,0}{#1}}%
\newcommand{\hlcom}[1]{\textcolor[rgb]{0.4,0.4,0.4}{\textit{#1}}}%
\newcommand{\hlopt}[1]{\textcolor[rgb]{0,0,0}{\textbf{#1}}}%
\newcommand{\hlstd}[1]{\textcolor[rgb]{0,0,0}{#1}}%
\newcommand{\hlkwa}[1]{\textcolor[rgb]{0,0,0}{\textbf{#1}}}%
\newcommand{\hlkwb}[1]{\textcolor[rgb]{0,0,0}{\textbf{#1}}}%
\newcommand{\hlkwc}[1]{\textcolor[rgb]{0,0,0}{\textbf{#1}}}%
\newcommand{\hlkwd}[1]{\textcolor[rgb]{0,0,0}{\textbf{#1}}}%

\usepackage{framed}
\makeatletter
\newenvironment{kframe}{%
 \def\at@end@of@kframe{}%
 \ifinner\ifhmode%
  \def\at@end@of@kframe{\end{minipage}}%
  \begin{minipage}{\columnwidth}%
 \fi\fi%
 \def\FrameCommand##1{\hskip\@totalleftmargin \hskip-\fboxsep
 \colorbox{shadecolor}{##1}\hskip-\fboxsep
     % There is no \\@totalrightmargin, so:
     \hskip-\linewidth \hskip-\@totalleftmargin \hskip\columnwidth}%
 \MakeFramed {\advance\hsize-\width
   \@totalleftmargin\z@ \linewidth\hsize
   \@setminipage}}%
 {\par\unskip\endMakeFramed%
 \at@end@of@kframe}
\makeatother

\definecolor{shadecolor}{rgb}{.97, .97, .97}
\definecolor{messagecolor}{rgb}{0, 0, 0}
\definecolor{warningcolor}{rgb}{1, 0, 1}
\definecolor{errorcolor}{rgb}{1, 0, 0}
\newenvironment{knitrout}{}{} % an empty environment to be redefined in TeX

\usepackage{alltt}

%%----------------------------------------------------------------------
%% opções comuns
\usepackage[brazilian]{babel}
\usepackage[utf8]{inputenc}
\usepackage[T1]{fontenc}
\usepackage{textcomp}
%\usepackage[margin=2cm]{geometry}
\usepackage{indentfirst}
\usepackage{fancybox}
%\usepackage[usenames,dvipsnames]{color}
\usepackage{amsmath,amsfonts,amssymb,amsthm}
\usepackage{lscape}
\usepackage{natbib}
\setlength{\bibsep}{0.0pt}
\usepackage{url}
\usepackage{multicol}
\usepackage{multirow}
\usepackage[final]{pdfpages}
\usepackage{setspace}
\usepackage{paralist} % enumitem, compactitem
  \plitemsep=2pt
%%----------------------------------------------------------------------

%%----------------------------------------------------------------------
%% FLOATS: graficos e tabelas
\usepackage{graphicx}
\usepackage{float} % fornece a opção [H] para floats
\usepackage{longtable}
\usepackage{supertabular}
%% captions e headings em sans-serif
\usepackage[font={sf},labelfont={sf,bf}]{caption}
\usepackage{subcaption}
\renewcommand{\thesubfigure}{\Alph{subfigure}}
\usepackage{titlesec}
\titleformat*{\section}{\normalsize\bfseries\sffamily}
\titleformat*{\subsection}{\normalsize\bfseries\sffamily}
\titleformat*{\subsubsection}{\normalsize\bfseries\sffamily}
\titleformat*{\paragraph}{\normalsize\bfseries\sffamily}
\titleformat*{\subparagraph}{\normalsize\bfseries\sffamily}
\theoremstyle{definition}
\newtheorem*{mydef}{Definição}
%%----------------------------------------------------------------------

%%----------------------------------------------------------------------
%% definiçoes de hyperref e xcolor
\usepackage{hyperref}
\usepackage{xcolor}
%%----------------------------------------------------------------------

%%----------------------------------------------------------------------
%% FONTES

%% micro-tipografia
\usepackage[protrusion=true,expansion=true]{microtype}
%% Bitstream Charter with mathdesign
\usepackage{lmodern} % sans-serif: Latin Modern
\usepackage[charter]{mathdesign} % serif: Bitstream Charter
\usepackage[scaled]{beramono} % truetype: Bistream Vera Sans Mono
\usepackage[scaled]{helvet}
%\usepackage{inconsolata}


%\usepackage[sf]{titlesec}
%%----------------------------------------------------------------------

%%----------------------------------------------------------------------
%% hifenização
\usepackage[htt]{hyphenat} % permite hifenizar texttt. Ao inves disso
% pode usar \allowbreak no ponto qu quiser quebrar dentro do texttt
\hyphenation{con-si-de-ra-ção pes-que-i-ros pes-que-i-ra se-gui-do-ras
  di-fe-ren-tes pla-ni-lha pla-ni-lhão re-fe-ren-te con-ta-gem
  em-bar-ques qua-li-da-de a-le-a-to-ri-za-dos}
%%----------------------------------------------------------------------

%%----------------------------------------------------------------------
%% comandos customizados
\usepackage{xspace} % lida com os espaços depois dos comandos
\providecommand{\eg}{\textit{e.g.}\xspace}
\providecommand{\ie}{\textit{i.e.}\xspace}
\providecommand{\R}{\textsf{R}\xspace}
\newcommand{\mb}[1]{\mathbf{#1}}
\newcommand{\bs}[1]{\boldsymbol{#1}}
\providecommand{\E}{\text{E}}
\providecommand{\Var}{\text{Var}}
\providecommand{\logit}{\text{logit}}
%% Para alterar o titulo do thebibliography
\addto\captionsbrazilian{%
  \renewcommand{\refname}{Bibliografia}
}
%%----------------------------------------------------------------------

%%----------------------------------------------------------------------
%% Comandos para deixar o texto mais compacto
\usepackage{marginnote}
\usepackage[top=1cm, bottom=1cm, inner=1cm, outer=1cm,nohead, nofoot, heightrounded, marginparsep=.05cm]{geometry}
\setlength{\parindent}{0pt}
%%----------------------------------------------------------------------
\IfFileExists{upquote.sty}{\usepackage{upquote}}{}
\begin{document}

\reversemarginpar % para colocar a marginnote a esquerda





\hrule
\vspace{0.3cm}

\begin{minipage}[c]{.85\textwidth}
  Estatística II --- CE003 \\
  Prof. Fernando de Pol Mayer --- Departamento de Estatística --- DEST \\
  Exercícios: probabilidade \\
  Nome:   \hfill GRR: \hspace{2cm}
\end{minipage}\hfill
\begin{minipage}[c]{.15\textwidth}
\flushright
\includegraphics[width=2.2cm]{../img/ufpr-logo.png}
\end{minipage}

\vspace{0.3cm}
\hrule
\vspace{0.3cm}
%%----------------------------------------------------------------------

\begin{compactenum} % Magalhaes, pg 40
\item Para cada um dos eventos abaixo, escreva o espaço amostral
  correspondente e conte seus elementos:
  \begin{compactenum}
  \item Uma moeda é lançada duas vezes e observam-se as faces obtidas.
  \item Um dado é lançado duas vezes e a ocorrência de face par ou ímpar
    é observada.
  \item Uma urna contém 10 bolas azuis e 10 vernelhas. Três bolas são
    selecionadas ao acaso, com reposição, e as cores são anotadas.
  \item Dois dados são lançados simultaneamente e estamos interessados
    na soma das faces observadas.
  \item Em uma cidade famílias com 3 crianças são selecionadas ao acaso,
    anotando-se o sexo de cada uma, de acordo com a idade.
  \item Uma máquina produz 20 peças por hora, escolhe-se um instante
    qualquer e observa-se o número de defeituosas na próxima hora.
  \item Uma moeda é lançada consecutivamente até o aparecimento da
    primeira cara.
    % A partir daqui: Bussab, pg 108, cap 5
  \item Mede-se a duração de lâmpadas, deixando-as acesas até que se
    queimem.
  \item Um fichário com 10 nomes contém 3 nomes de mulheres.
    Seleciona-se ficha após ficha, até o último nome de mulher ser
    selecionado, e anota-se o número de fichas selecionadas.
  \item Uma moeda é lançada consecutivamente até o aparecimento da
    primeira cara e anota-se o número de lançamentos.
  \item De um grupo de 5 pessoas ${A, B, C, D, E}$, soteiam-se duas, uma
    após a outra, com reposição, e anota-se a configuração obtida.
  \item Mesmo enunciado anterior, mas sem reposição.
  \item Mesmo enunciado anterior, mas as duas selecionadas
    simultaneamente.
  \end{compactenum}

\vspace{0.3cm}
\hrule
\vspace{0.3cm}

% Bussab, pg 107, cap. 5
\item Uma urna contém duas bolas brancas e três bolas vermelhas.
  Retira-se uma bola ao acaso da urna. Se for branca lança-se uma moeda.
  Se for vermelha, ela é devolvida à urna e retira-se outra. Dê um
  espaço amostral para o experimento.

\vspace{0.3cm}
\hrule
\vspace{0.3cm}

% Ross, pg 71
\item Uma caixa contém 3 bolas de gude: 1 vermelha, 1 azul e 1
  branca. Considere um experimento que consiste em retirar duas bolas de
  gude desta caixa. Descreva o espaço amostral quando:
  \begin{compactenum}
  \item houver reposição da primeira bola retirada da caixa
  \item não houver reposição da primeira bola retirada da caixa
  \end{compactenum}

\vspace{0.3cm}
\hrule
\vspace{0.3cm}

% Montgomery, pg 19, cap 2
\item Uma balança digital é usada para fornecer pesos em gramas. Seja
  $A$ o evento em que um peso excede 11 gramas. Seja $B$ o evento em que
  um peso é menor que ou igual a 15 gramas, e seja $C$ o evento em que
  um peso é maior ou igual a 8 gramas e menor que 12 gramas. Descreva os
  seguintes eventos: \\
  \begin{inparaenum}
  \item $\Omega$ \,
  \item $A \cup B$ \,
  \item $A \cap B$ \,
  \item $A^{c}$ \,
  \item $A \cup B \cup C$ \,
  \item $(A \cup C)^{c}$ \,
  \item $A \cap B \cap C$ \,
  \item $B^{c} \cap C$ \,
  \item $A \cup (B \cap C)$
  \end{inparaenum}

\vspace{0.3cm}
\hrule
\vspace{0.3cm}

% Montgomery, pg 19, cap 2
\item Na fotossíntese, a qualidade da luz se refere aos comprimentos de
  onda de luz que são importantes. O comprimento de onda de uma amostra
  de radiações fotossinteticamente ativas (RFA) é medido em nanômetro. A
  faixa do vermelho é 675-700 nm, e a faixa do azul é 450-500 nm. Seja
  $A$ o evento em que RFA ocorre na faixa do vermelho, e $B$ o evento em
  que RFA ocorre na faixa do azul. Descreva o espaço amostral e indique
  cada um dos seguintes eventos: \\
  \begin{inparaenum}
  \item $A$ \,
  \item $B$ \,
  \item $A \cap B$ \,
  \item $A \cup B$
  \end{inparaenum}

\vspace{0.3cm}
\hrule
\vspace{0.3cm}

% Montgomery, pg 19, cap 2
\item Em replicação controlada, células são replicadas em um período de
  dois dias. DNA recém-sintetizado não pode ser replicado novamente até
  que a mtose seja completa. Dois mecanismos de controle foram
  identificados: um positivo e um negativo. Suponha que uma replicação
  seja observada em três células. Seja $A$ o evento em que todas as
  células são identificadas como positivas, e $B$ o evento em que todas
  as células são negativas. Descreva o espaço amostral e indique cada um
  dos seguintes eventos: \\
  \begin{inparaenum}
  \item $A$ \,
  \item $B$ \,
  \item $A \cap B$ \,
  \item $A \cup B$
  \end{inparaenum}

\vspace{0.3cm}
\hrule
\vspace{0.3cm}

% Bussab, pg 113, cap. 5
\item Considere o lançamento de dois dados. Considere os eventos $A$ =
  ``soma dos números obtidos igual a 9'', e $B$ = ``número no primeiro
  dado maior ou igual a 4''.
  \begin{compactenum}
  \item Enumere os elementos de $A$ e $B$.
  \item Obtenha $A \cup B$, $A \cap B$, e $A^{c}$.
  \item Obtenha todas as probabilidades dos eventos acima.
  \end{compactenum}

\vspace{0.3cm}
\hrule
\vspace{0.3cm}

% Bussab, pg 126, cap. 5
\item Suponhamos que 10000 bilhetes sejam vendidos em uma loteria e 5000
  em outra, cada uma tendo apenas um ganhador. Um homem tem 100 bilhetes
  de cada. Qual a probabilidade de que:
  \begin{compactenum}
  \item Ele ganhe exatamente um prêmio?
  \item Ele ganhe alguma coisa?
  \end{compactenum}

\vspace{0.3cm}
\hrule
\vspace{0.3cm}

% Bussab, pg 126, cap. 5
\item Um grupo de 12 homens e 8 mulheres concorre a três prêmios através
  de um sorteio, sem reposição de seus nomes. Qual a probabilidade de:
  \begin{compactenum}
  \item Nenhum homem ser sorteado?
  \item Um prêmio ser ganho por homem?
  \item Dois homens serem premiados?
  \end{compactenum}

\vspace{0.3cm}
\hrule
\vspace{0.3cm}

% Ross pg 71
\item Suponha que $A$ e $B$ sejam eventos mutuamente exclusivos (ou
  disjuntos) para os quais $P(A) = 0,3$ e $P(B) = 0,5$. Qual é a
  probabilidade de que:
  \begin{compactenum}
  \item $A$ ou $B$ ocorra?
  \item $A$ ocorra mas $B$ não ocorra?
  \item $A$ e $B$ ocorram?
  \end{compactenum}

\vspace{0.3cm}
\hrule
\vspace{0.3cm}

% Montgomery, pg 19, cap 2
\item Discos de plático de policarbonato, provenientes de um fornecedor,
  são analisados com relação à resistência a arranhões e a choque. Os
  resultados de 100 discos estão resumidos a seguir:
  \begin{table}[!h]
    \centering
    \begin{tabular}{lcc}
      \hline
      \multirow{2}{*}{\textbf{Res. a arranhões}}
      & \multicolumn{2}{l}{\textbf{Res. a choques}} \\
      \cline{2-3}
                & Alta         & Baixa       \\
      \hline
      Alta      & 70           & 9           \\
      Baixa     & 16           & 5           \\
      \hline
    \end{tabular}
  \end{table}

  Seja $A$ o evento em que um disco tem alta resistência a choque e $B$
  o evento em que um disco tem alta resistência a arranhões. Com isso:
  \begin{compactenum}
  \item Determine o número de discos em $A \cap B$, $A^{c}$, e $A \cup B$.
  \item Se um disco for selecionado aleatoriamente, determine as
    seguintes probabilidades: \\
    \begin{inparaenum}
    \item $P(A)$ \,
    \item $P(B)$ \,
    \item $P(A^{c})$ \,
    \item $P(A \cap B)$ \,
    \item $P(A \cup B)$ \,
    \item $P(A^{c} \cup B)$
    \end{inparaenum}
  \item Se um disco for selecionado ao acaso, qual será a probabilidade
    de sua resistência a arranhões ser alta e de sua resistência a
    choque ser alta?
  \item Se um disco for selecionado ao acaso, qual será a probabilidade
    de sua resistência a arranhões ser alta ou de sua resistência a
    choque ser alta?
  \item Os eventos $A$ e $B$ são mutuamente exclusivos?
  \end{compactenum}

\vspace{0.3cm}
\hrule
\vspace{0.3cm}

% Montgomery, pg 20, cap 2
\item O tempo de enchimento de um reator é medido em minutos (e frações
  de minutos). Seja $\Omega = \mathbb{R}^{+}$. Defina os aventos $A$ e
  $B$ como segue:
  \begin{equation*}
    A = \{x : x \leq 72,5\} \qquad \text{e} \qquad B = \{x : x > 52,5\}
  \end{equation*}
  Descreva cada um dos seguintes eventos: \\
  \begin{inparaenum}
  \item $A^{c}$ \,
  \item $B^{c}$ \,
  \item $A \cap B$ \,
  \item $A \cup B$
  \end{inparaenum}

\vspace{0.3cm}
\hrule
\vspace{0.3cm}

% Montgomery, pg 20, cap 2
\item Uma amostra de dois itens é selecionada sem reposição a partir de
  uma batelada. Descreva o espaço amostral (ordenado) para cada uma das
  seguintes bateladas:
  \begin{compactenum}
  \item A batelada contém os itens $\{a, b, c, d\}$
  \item A batelada contém os itens $\{a, b, c, d, e, f, g\}$
  \item A batelada contém 4 itens defeituosos e 20 itens bons
  \item A batelada contém 1 item defeituoso e 20 itens bons
  \end{compactenum}

\vspace{0.3cm}
\hrule
\vspace{0.3cm}

% Montgomery, pg 20, cap 2
\item Uma amostra de duas placas de circuito impresso é selecionada sem
  reposição a partir de uma batelada. Descreva o espaço amostral
  (ordenado) para cada uma das seguintes bateladas:
  \begin{compactenum}
  \item A batelada contém 90 placas que são não defeituosas, 8 placas
    com pequenos defeitos, e 2 placas com grandes defeitos.
  \item A batelada contém 90 placas que são não defeituosas, 8 placas
    com pequenos defeitos, e 1 placa com grandes defeitos.
  \end{compactenum}

\vspace{0.3cm}
\hrule
\vspace{0.3cm}

% Ross pg 71
\item Uma loja aceita cartões de crédito American Express ou Visa. Um
  total de 24\% de seus consumidores possui um cartão American Express,
  61\% possuem Visa, e 11\% possuem ambos. Que percentual desses
  consumidores possui um cartão aceito pelo estabelecimento?

\vspace{0.3cm}
\hrule
\vspace{0.3cm}

% magalhaes pg 41
\item Sejam $A$ e $B$ dois eventos em um espaço amostral, tais que $P(A)
  = 0,2$, $P(B) = p$, $P(A \cup B) = 0,5$, e $P(A \cap B) =
  0,1$. Determine o valor de $p$.

\vspace{0.3cm}
\hrule
\vspace{0.3cm}

% Magalhaes, pg 41
\item Uma Universidade tem 10000 alunos, dos quais 4000 são considerados
  esportistas. Temos ainda que 500 alunos são do curso de Economia, 700
  são de Administração, 100 são esportistas e da Economia, e 200 são
  esportistas e da Administração. (Os restantes dos alunos que não se
  encaixam em nenhum curso podem ser colocados na categoria
  ``outros''). Monte a tabela de contingência, e calcule as
  probabilidades de um aluno selecionado ao acaso:
  \begin{compactenum}
  \item Ser esportista
  \item Ser esportista e aluno da Administração
  \item Não ser da Economia nem da Administração
  \item Não ser esportista, e ser de outros cursos
  \item Ser esportista, dado que faz Economia
  \item Fazer Administração, dado que não é esportista
  \item Não ser esportista, dado que não faz nem Economia, nem
    Administração
  \item Fazer outros cursos, dado que é esportista
  \end{compactenum}

\vspace{0.3cm}
\hrule
\vspace{0.3cm}

% Kazmier, pg 78
\item Suponhamos que um um aluno estime que sua probabilidade de receber um
  conceito final ``A'' em Estatística é 0,6, e a probabilidade de um
  ``B'' é de 0,4 (ele não considera receber menos do que isso!). Com isso:
  \begin{compactenum}
  \item Os eventos ``receber A'' e ``receber B'' são mutuamente
    exclusivos? Por que?
  \item Determine a probabilidade condicional de que obtenha um ``B'',
    dado que de fato tenha recebido um ``A''.
  \item Verifique se os eventos ``receber A'' e ``receber B'' são
    independentes.
  \end{compactenum}

\vspace{0.3cm}
\hrule
\vspace{0.3cm}

% morettin, pg 23
\item Sejam $A$ e $B$ eventos tais que $P(A) = 0,2$, $P(B) = p$, $P(A
  \cup B) = 0,6$. Calcular o valor de $p$ considerando $A$ e $B$
  \begin{compactenum}
  \item mutuamente exclusivos
  \item independentes
  \end{compactenum}

\vspace{0.3cm}
\hrule
\vspace{0.3cm}

% morettin, pg 40
\item Uma urna contém 10 bolas verdes e 6 azuis. Tiram-se duas bolas ao
  acaso, sem reposição. Qual a probabilidade de que as duas bolas:
  \begin{compactenum}
  \item sejam verdes?
  \item sejam da mesma cor?
  \item sejam de cores diferentes?
  \end{compactenum}

\vspace{0.3cm}
\hrule
\vspace{0.3cm}

% kazmier, pg 77
\item De 100 pessoas que solicitaram emprego de programador de
  computadores, durante o ano passado, em uma grande empresa, 40
  possuíam experiência anterior ($E$), e 30 possuíam certificado
  profissional ($C$). Vinte dos candidatos possuíam tanto experiência
  anterior como certificado profissional e foram incluídos nas contagens
  dos dois grupos.
  \begin{compactenum}
  \item Elaborar um diagrama de Venn para representar estes eventos.
  \item Qual a probabilidade de que um candidato escolhido ao acaso
    tenha experiência ou certificado (ou ambos)?
  \item Qual a probabilidade de que um candidato escolhido ao acaso
    tenha experiência ou certificado, mas não ambos?
  \item Determinar a probabilidade condicional de que um candidato
    escolhido ao acaso tenha um certificado, dado que ele tenha alguma
    experiência anterior.
  \item Verificar se os eventos $E$ e $C$ são independentes.
  \end{compactenum}

\vspace{0.3cm}
\hrule
\vspace{0.3cm}

% kazmier, pg 78
\item Em geral, a probabilidade de que um possível cliente faça uma
  compra quando procurado por um vendedor é de 0,4. Se um vendedor
  seleciona do arquivo, aleatoriamente, três clientes, e faz contato com
  os mesmos, qual a probabilidade de que os três façam compras?

\vspace{0.3cm}
\hrule
\vspace{0.3cm}

\end{compactenum}

\end{document}

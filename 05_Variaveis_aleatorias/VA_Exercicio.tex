\documentclass[a4paper,11pt,fleqn]{article}\usepackage[]{graphicx}\usepackage[]{color}
%% maxwidth is the original width if it is less than linewidth
%% otherwise use linewidth (to make sure the graphics do not exceed the margin)
\makeatletter
\def\maxwidth{ %
  \ifdim\Gin@nat@width>\linewidth
    \linewidth
  \else
    \Gin@nat@width
  \fi
}
\makeatother

\definecolor{fgcolor}{rgb}{0, 0, 0}
\newcommand{\hlnum}[1]{\textcolor[rgb]{0,0,0}{#1}}%
\newcommand{\hlstr}[1]{\textcolor[rgb]{0,0,0}{#1}}%
\newcommand{\hlcom}[1]{\textcolor[rgb]{0.4,0.4,0.4}{\textit{#1}}}%
\newcommand{\hlopt}[1]{\textcolor[rgb]{0,0,0}{\textbf{#1}}}%
\newcommand{\hlstd}[1]{\textcolor[rgb]{0,0,0}{#1}}%
\newcommand{\hlkwa}[1]{\textcolor[rgb]{0,0,0}{\textbf{#1}}}%
\newcommand{\hlkwb}[1]{\textcolor[rgb]{0,0,0}{\textbf{#1}}}%
\newcommand{\hlkwc}[1]{\textcolor[rgb]{0,0,0}{\textbf{#1}}}%
\newcommand{\hlkwd}[1]{\textcolor[rgb]{0,0,0}{\textbf{#1}}}%

\usepackage{framed}
\makeatletter
\newenvironment{kframe}{%
 \def\at@end@of@kframe{}%
 \ifinner\ifhmode%
  \def\at@end@of@kframe{\end{minipage}}%
  \begin{minipage}{\columnwidth}%
 \fi\fi%
 \def\FrameCommand##1{\hskip\@totalleftmargin \hskip-\fboxsep
 \colorbox{shadecolor}{##1}\hskip-\fboxsep
     % There is no \\@totalrightmargin, so:
     \hskip-\linewidth \hskip-\@totalleftmargin \hskip\columnwidth}%
 \MakeFramed {\advance\hsize-\width
   \@totalleftmargin\z@ \linewidth\hsize
   \@setminipage}}%
 {\par\unskip\endMakeFramed%
 \at@end@of@kframe}
\makeatother

\definecolor{shadecolor}{rgb}{.97, .97, .97}
\definecolor{messagecolor}{rgb}{0, 0, 0}
\definecolor{warningcolor}{rgb}{1, 0, 1}
\definecolor{errorcolor}{rgb}{1, 0, 0}
\newenvironment{knitrout}{}{} % an empty environment to be redefined in TeX

\usepackage{alltt}

%%----------------------------------------------------------------------
%% opções comuns
\usepackage[brazilian]{babel}
\usepackage[utf8]{inputenc}
\usepackage[T1]{fontenc}
\usepackage{textcomp}
%\usepackage[margin=2cm]{geometry}
\usepackage{indentfirst}
\usepackage{fancybox}
%\usepackage[usenames,dvipsnames]{color}
\usepackage{amsmath,amsfonts,amssymb,amsthm}
\usepackage{lscape}
\usepackage{natbib}
\setlength{\bibsep}{0.0pt}
\usepackage{url}
\usepackage{multicol}
\usepackage{multirow}
\usepackage[final]{pdfpages}
\usepackage{setspace}
\usepackage{paralist} % enumitem, compactitem
  \plitemsep=2pt
%%----------------------------------------------------------------------

%%----------------------------------------------------------------------
%% FLOATS: graficos e tabelas
\usepackage{graphicx}
\usepackage{float} % fornece a opção [H] para floats
\usepackage{longtable}
\usepackage{supertabular}
%% captions e headings em sans-serif
\usepackage[font={sf},labelfont={sf,bf}]{caption}
\usepackage{subcaption}
\renewcommand{\thesubfigure}{\Alph{subfigure}}
\usepackage{titlesec}
\titleformat*{\section}{\normalsize\bfseries\sffamily}
\titleformat*{\subsection}{\normalsize\bfseries\sffamily}
\titleformat*{\subsubsection}{\normalsize\bfseries\sffamily}
\titleformat*{\paragraph}{\normalsize\bfseries\sffamily}
\titleformat*{\subparagraph}{\normalsize\bfseries\sffamily}
\theoremstyle{definition}
\newtheorem*{mydef}{Definição}
%%----------------------------------------------------------------------

%%----------------------------------------------------------------------
%% definiçoes de hyperref e xcolor
\usepackage{hyperref}
\usepackage{xcolor}
%%----------------------------------------------------------------------

%%----------------------------------------------------------------------
%% FONTES

%% micro-tipografia
\usepackage[protrusion=true,expansion=true]{microtype}
%% Bitstream Charter with mathdesign
\usepackage{lmodern} % sans-serif: Latin Modern
\usepackage[charter]{mathdesign} % serif: Bitstream Charter
\usepackage[scaled]{beramono} % truetype: Bistream Vera Sans Mono
\usepackage[scaled]{helvet}
%\usepackage{inconsolata}


%\usepackage[sf]{titlesec}
%%----------------------------------------------------------------------

%%----------------------------------------------------------------------
%% hifenização
\usepackage[htt]{hyphenat} % permite hifenizar texttt. Ao inves disso
% pode usar \allowbreak no ponto qu quiser quebrar dentro do texttt
\hyphenation{con-si-de-ra-ção pes-que-i-ros pes-que-i-ra se-gui-do-ras
  di-fe-ren-tes pla-ni-lha pla-ni-lhão re-fe-ren-te con-ta-gem
  em-bar-ques qua-li-da-de a-le-a-to-ri-za-dos}
%%----------------------------------------------------------------------

%%----------------------------------------------------------------------
%% comandos customizados
\usepackage{xspace} % lida com os espaços depois dos comandos
\providecommand{\eg}{\textit{e.g.}\xspace}
\providecommand{\ie}{\textit{i.e.}\xspace}
\providecommand{\R}{\textsf{R}\xspace}
\newcommand{\mb}[1]{\mathbf{#1}}
\newcommand{\bs}[1]{\boldsymbol{#1}}
\providecommand{\E}{\text{E}}
\providecommand{\Var}{\text{Var}}
\providecommand{\logit}{\text{logit}}
%% Para alterar o titulo do thebibliography
\addto\captionsbrazilian{%
  \renewcommand{\refname}{Bibliografia}
}
%%----------------------------------------------------------------------

%%----------------------------------------------------------------------
%% Comandos para deixar o texto mais compacto
\usepackage{marginnote}
\usepackage[top=1cm, bottom=1cm, inner=1cm, outer=1cm,nohead, nofoot, heightrounded, marginparsep=.05cm]{geometry}
\setlength{\parindent}{0pt}
%%----------------------------------------------------------------------
\IfFileExists{upquote.sty}{\usepackage{upquote}}{}
\begin{document}

\reversemarginpar % para colocar a marginnote a esquerda





\hrule
\vspace{0.3cm}

\begin{minipage}[c]{.85\textwidth}
  Estatística II --- CE003 \\
  Prof. Fernando de Pol Mayer --- Departamento de Estatística --- DEST \\
  Exercício: variáveis aleatórias \\
  Nome:   \hfill GRR: \hspace{2cm}
\end{minipage}\hfill
\begin{minipage}[c]{.15\textwidth}
\flushright
\includegraphics[width=2.2cm]{../img/ufpr-logo.png}
\end{minipage}

\vspace{0.3cm}
\hrule
\vspace{0.3cm}
%%----------------------------------------------------------------------

%% Bussab e Morettin, Cap 6, pg 129

\begin{compactenum}[1.]
\item Um empresário pretende estabelecr uma firma para montagem de um
  produto composto de uma esfera e um cilindro. As partes são adquiridas
  em fábricas diferentes (A e B), e a montagem consistirá em juntar as
  duas partes e pintá-las. O produto acabado deve ter o comprimento
  (definido pelo cilindro), e a espessura (definido pela esfera) dentro
  de certos limites, e isso só poderá ser verificado após a
  montagem. Para estudar a viabilidade de seu empreendimento, o
  empresário quer ter uma ideia da distribuição do lucro por peça
  montada.

  Sabe-se que cada componente pode ser classificado como bom, longo ou
  curto, conforme sua medida esteja dentro de uma especificação, maior
  ou menor que a especificada, respectivamente. Além disso, foram
  obtidos dos fabricantes o preço de cada componente (R\$ 5,00), e as
  probabilidades de produção de cada componente com as características
  bom, longo e curto. Esses valores estão na tabela abaixo.

  \begin{table}[htbp]
    \begin{center}
      \begin{tabular}{l|c|c}
        \hline
        \multicolumn{1}{c|}{Produto} & Fábrica A (Cilindro) & Fábrica B (Esfera) \\ \hline
        Dentro das especificações … bom (B) & 0,8 & 0,7 \\
        Maior que as especificações … longo (L) & 0,1 & 0,2 \\
        Menor que as especificações … curto (C) & 0,1 & 0,1 \\ \hline
      \end{tabular}
    \end{center}
  \end{table}

  O empresário assume, com base em produtos similares no mercado, que o
  preço final de venda de cada unidade do produto (esfera e cilindro)
  deve ser de R\$ 25,00. Se o produto final apresentar algum componente
  com a característica C (curto), ele será irrecuperável e o conjunto
  será vendido como sucata ao preço de R\$ 5,00. Cada componente longo
  poderá ser recuperado a um custo adicional de R\$ 5,00. Com isso:
  \begin{compactenum}
  \item Defina o espaço amostral para a montagem dos conjuntos.
  \item Obtenha as probabilidades associadas a cada ponto do espaço
    amostral, ou seja, a probabilidade de se obter cada conjunto
    possível.
  \item Definindo a variável aleatória $X$ = ``lucro por conjunto
    montado'', obtenha o lucro por montagem, considerando as informações
    sobre preços dadas no enunciado.
  \item Obtenha a distribuição de probabilidade da variável aleatória
    $X$.
  \item Se considerarmos a variável aleatória $Y$ = ``custo de
    recuperação de cada conjunto produzido'', obtenha a distribuição de
    probabilidade de $Y$.
  \item Qual é o lucro médio por conjunto montado que o empresário
    espera conseguir? Calcule também a variância e o desvio-padrão do
    lucro.
  \item Suponha que todos os preços determinados pelo empresário estão
    errados. Na realidade, todos os valores deveriam ser duplicados,
    isto é, custos e preços de venda. Com isso, temos uma nova variável
    aleatória $Z = 2X$. Determine a média, a variância e o desvio-padrão
    desta nova variável aleatória $Z$.
  \end{compactenum}


\end{compactenum}

\vspace{0.3cm}
\hrule
\vspace{0.3cm}



\end{document}

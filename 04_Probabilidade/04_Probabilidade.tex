\pdfminorversion=4
\documentclass[10pt]{beamer}\usepackage[]{graphicx}\usepackage[]{color}
%% maxwidth is the original width if it is less than linewidth
%% otherwise use linewidth (to make sure the graphics do not exceed the margin)
\makeatletter
\def\maxwidth{ %
  \ifdim\Gin@nat@width>\linewidth
    \linewidth
  \else
    \Gin@nat@width
  \fi
}
\makeatother

\definecolor{fgcolor}{rgb}{0.345, 0.345, 0.345}
\newcommand{\hlnum}[1]{\textcolor[rgb]{0.686,0.059,0.569}{#1}}%
\newcommand{\hlstr}[1]{\textcolor[rgb]{0.192,0.494,0.8}{#1}}%
\newcommand{\hlcom}[1]{\textcolor[rgb]{0.678,0.584,0.686}{\textit{#1}}}%
\newcommand{\hlopt}[1]{\textcolor[rgb]{0,0,0}{#1}}%
\newcommand{\hlstd}[1]{\textcolor[rgb]{0.345,0.345,0.345}{#1}}%
\newcommand{\hlkwa}[1]{\textcolor[rgb]{0.161,0.373,0.58}{\textbf{#1}}}%
\newcommand{\hlkwb}[1]{\textcolor[rgb]{0.69,0.353,0.396}{#1}}%
\newcommand{\hlkwc}[1]{\textcolor[rgb]{0.333,0.667,0.333}{#1}}%
\newcommand{\hlkwd}[1]{\textcolor[rgb]{0.737,0.353,0.396}{\textbf{#1}}}%

\usepackage{framed}
\makeatletter
\newenvironment{kframe}{%
 \def\at@end@of@kframe{}%
 \ifinner\ifhmode%
  \def\at@end@of@kframe{\end{minipage}}%
  \begin{minipage}{\columnwidth}%
 \fi\fi%
 \def\FrameCommand##1{\hskip\@totalleftmargin \hskip-\fboxsep
 \colorbox{shadecolor}{##1}\hskip-\fboxsep
     % There is no \\@totalrightmargin, so:
     \hskip-\linewidth \hskip-\@totalleftmargin \hskip\columnwidth}%
 \MakeFramed {\advance\hsize-\width
   \@totalleftmargin\z@ \linewidth\hsize
   \@setminipage}}%
 {\par\unskip\endMakeFramed%
 \at@end@of@kframe}
\makeatother

\definecolor{shadecolor}{rgb}{.97, .97, .97}
\definecolor{messagecolor}{rgb}{0, 0, 0}
\definecolor{warningcolor}{rgb}{1, 0, 1}
\definecolor{errorcolor}{rgb}{1, 0, 0}
\newenvironment{knitrout}{}{} % an empty environment to be redefined in TeX

\usepackage{alltt}
%% O comando acima foi necessario porque o PDF nao abria no acrobat do
%% windows, dava o erro 131. Provavelmente devido as figuras em
%% PDF. Agora ele gera um PDF versao 1.4, ao inves da versao 1.5

\usetheme[compress]{PaloAlto}
\usecolortheme{sidebartab} % crane

\usepackage[brazilian]{babel}
\usepackage[T1]{fontenc}
\usepackage[utf8]{inputenc}
\usepackage{graphicx}
\usepackage{hyperref}
\usepackage[scaled]{beramono} % truetype: Bistream Vera Sans Mono
%\usepackage{inconsolata}
\usepackage{xfrac}
\usepackage{tikz}
\usepackage{xcolor}
\usepackage{multirow}
\usepackage{multicol}

\setbeamertemplate{footline}[frame number] % mostra o numero dos slides
\setbeamertemplate{navigation symbols}{} % retira a barra de navegacao

\usepackage{xspace}
\providecommand{\eg}{\textit{e.g.}\xspace}
\providecommand{\ie}{\textit{i.e.}\xspace}
\providecommand{\R}{\textsf{R}\xspace}
\newcommand{\mb}[1]{\mathbf{#1}}
\newcommand{\bs}[1]{\boldsymbol{#1}}
\providecommand{\E}{\text{E}}
\providecommand{\Var}{\text{Var}}
\theoremstyle{definition}
\newtheorem*{mydef}{Definição}
\newtheorem*{mythm}{Teorema}

\title{Probabilidade}
\author[]{Fernando de Pol Mayer}
\institute[UFPR]{Laboratório de Estatística e Geoinformação (LEG) \\
  Departamento de Estatística (DEST) \\
  Universidade Federal do Paraná (UFPR)}
\date{}
\logo{\includegraphics[width=1.6cm]{../img/ufpr-logo.png}}
\titlegraphic{\includegraphics[width=1cm]{../img/CC_by-nc-sa_88x31.png}\\
  \tiny
  \href{https://creativecommons.org/licenses/by-nc-sa/4.0/deed.pt_BR}{Este
    conteúdo está disponível por meio da Licença Creative Commons 4.0
    (Atribuição/NãoComercial/PartilhaIgual)}}

\AtBeginSection[]
{
  \begin{frame}
    \frametitle{Plano de aula}
    \tableofcontents[currentsection]
  \end{frame}
}

\AtBeginSubsection[]
{
  \begin{frame}
    \frametitle{Plano de aula}
    \tableofcontents[currentsection,currentsubsection]
  \end{frame}
}
\IfFileExists{upquote.sty}{\usepackage{upquote}}{}
\begin{document}





\begin{frame}
\maketitle
%\titlepage
\end{frame}

\begin{frame}{Sumário}
\tableofcontents
\end{frame}

\section[Introdução]{Introdução}

\begin{frame}{Introdução}
  \begin{quote}
    ``A razão do número de todos os casos favoráveis à um acontecimento,
    para o de todos os casos possíveis é a probabilidade buscada, a qual
    é portanto uma fração [\ldots]''
  \end{quote}
  \vspace{1em}
  \begin{quote}
    ``A teoria da probabilidade nada mais é do que o senso comum
    reduzido à cálculo.''
  \end{quote}
  \vspace{1em}
  \begin{flushright}
    --- Pierre Simon Laplace\\
    \textbf{Ensaio Filosófico Sobre as Probabilidades (1812)}
  \end{flushright}
\end{frame}

\begin{frame}{Introdução}
A \textbf{Teoria das Probabilidades} é o ramo da Matemática que cria,
desenvolve e pesquisa \textbf{modelos} que podem ser utilizados para
estudar \textbf{experimentos ou fenômenos aleatórios} \\~\\
A \textbf{Inferência Estatística} é totalmente fundamentada na
\textbf{Teoria das Probabilidades} \\~\\
O \textbf{modelo} utilizado para estudar um fenômeno aleatório pode variar em
complexidade, mas todos eles possuem ingredientes básicos comuns.
\end{frame}

\begin{frame}{Tipos de experimentos}
  \begin{block}{Experimentos determinísticos}
    Dizemos que um experimento é determinístico quando repetido inúmeras
    vezes, \textbf{em condições semelhantes}, conduz a resultados
    \textit{essencialmente} idênticos. Ex.:
    \begin{itemize}
    \item Aceleração da gravidade
    \item Leis da Física e da Química
    \end{itemize}
  \end{block}
  \begin{block}{Experimentos aleatórios}
    Os experimentos que \textbf{repetidos sob as mesmas condições} geram
    resultados diferentes, são chamados de experimentos aleatórios. Ex.:
    \begin{itemize}
    \item Lançamento de uma moeda
    \item Lançamento de um dado
    \item Tempo de vida de um equipamento eletrônico
    \end{itemize}
  \end{block}
\end{frame}

\begin{frame}{Introdução}
  O objetivo é construir um modelo matemático para representar
  \textbf{experimentos aleatórios}. Isso o corre em duas etapas: \\~\\
  \begin{enumerate}
  \item Descrever o \textbf{conjunto} de resultados possíveis
  \item Atribuir \textit{pesos} a cada resultado, refletindo suas
    chances de ocorrência
  \end{enumerate}
\end{frame}

%% \begin{frame}{Exemplo}
%%   Exemplo do dado do bussab
%% \end{frame}

\section{Experimentos e eventos}

%% \begin{frame}{Experimentos e eventos}
%%   \begin{block}{Espaço amostral ($\Omega$)}
%%     Conjunto de todos os resultados possíveis de um experimento
%%     aleatório. O $\Omega$ pode conter um número finito ou infinito de
%%     pontos. Exemplo: \{cara,coroa\},\{1,2,3,4,5,6\}, $\mathbb{N}$,
%%     $\mathbb{R}$.
%%   \end{block}
%%   \begin{block}{Ponto amostral ($\omega$)}
%%    São elementos de um espaço amotral. Exemplo: $\omega_1=\text{cara}$,
%%    $\omega_2=\text{coroa}$.
%%   \end{block}
%%   \begin{block}{Evento}
%%     É um subconjunto de pontos do espaço amostral de um experimentos
%%     aleatório. Exemplo: A = ``sair face par'', B = ``sair face menor que
%%     3''.
%%   \end{block}
%%  \end{frame}

\begin{frame}{Experimentos e eventos}
Um experimento, que ao ser realizado sob as mesmas condições não
 produz os mesmos resultados, é denominado um \textbf{experimento
   aleatório}. Exemplo: lançamento de uma moeda, medir altura, \ldots
 \\~\\
 O conjunto de todos os possíveis resultados de um experimento
 aleatório é denominado \textbf{espaço amostral} ($\Omega$). Pode conter
 um número finito ou infinito de pontos. Exemplo: \{cara, coroa\},
 $\mathbb{R}$, \ldots  \\~\\
 Os elementos do espaço amostral (\textbf{pontos amostrais}) são
 denotados por $\omega$. Exemplo: $\omega_1 = \text{cara}$, $\omega_2 =
 \text{coroa}$. \\~\\
 Todo resultado ou \underline{subconjunto} de resultados de um
 experimento aleatório, é um \textbf{evento}. Exemplo: A = ``sair
 cara'', B = ``sair face par''.
\end{frame}

\begin{frame}{Exemplos}
  \begin{block}{}
    \begin{description}
    \item[Experimento] lançar o dado e observar o resultado da
      face.
    \item[Espaço amostral] $\Omega=\{1,2,3,4,5,6\}$.
    \item[Pontos amostrais] $\omega_1=\text{1}$, $\omega_2=\text{2}$,
      \ldots, $\omega_6=\text{6}$.
    \item[Eventos] A = ``sair face par'', B = $\{\omega: \omega\leq
      4\}$.
    \end{description}
  \end{block}
  \begin{block}{}
    \begin{description}
    \item[Experimento] retirar uma carta de um baralho de 54
      cartas.
    \item[Espaço amostral] $\Omega=\{\clubsuit A,\clubsuit
      2,\ldots,\heartsuit A,\ldots,\spadesuit A, \ldots,\diamondsuit J,
      \diamondsuit Q, \diamondsuit K\}$.
    \item[Pontos amostrais] $\omega_1=\clubsuit A$, $\omega_2=\clubsuit
      2$, \ldots, $\omega_{54}=\diamondsuit K$.
    \item[Eventos] A = ``sair um ás'', B = ``sair uma letra'', C =
      ``sair carta de $\clubsuit$''.
    \end{description}
  \end{block}
  \begin{block}{}
    \begin{description}
    \item[Experimento] pesar um fruto ao acaso
    \item[Espaço amostral] $\Omega=\mathbb{R}^{+}$.
    \item[Pontos amostrais] espaço amostral é infinito.
    \item[Eventos] A = ``peso menor que 50g'', B = $\{x: x\geq100\text{g}\}$.
    \end{description}
  \end{block}
  \normalsize
\end{frame}

\begin{frame}{Exemplos}
\begin{small}
  \textbf{Exemplo 1}: Considere um experimento em que você seleciona
  uma peça plástica moldada, e mede sua espessura.
  \begin{enumerate}
  \item Qual o espaço amostral?
  \item Se é sabido que as peças só podem variar entre 10 e 11 mm de
    espessura, qual o espaço amostral?
  \item Se o objetivo da análise for considerar apenas o fato de uma
    peça ter espessura baixa, média ou alta, qual o espaço amostral?
  \item Se o objetivo for considerar o fato de uma peça obedecer ou não
    às especificações, qual o espaço amostral?
  \end{enumerate}
  \textbf{Exemplo 2}: Duas peças plásticas são selecionadas e medidas.
  \begin{enumerate}
  \item Se o objetivo é verificar se cada peça obedece ou não às
    especificações, qual o espaço amostral?
  \item Se o objetivo for somente o número de peças não conformes na
    amostra, qual o espaço amostral?
  \item Considere que a espessura é medida até que se encontre a
    primeira peça fora das especificações. Qual o espaço amostral?
  \end{enumerate}
\end{small}
\end{frame}

\begin{frame}{Operações com eventos}
  Usamos a \textbf{Teoria dos conjuntos} para definir operações com
  eventos
  \begin{block}{}
\begin{description}
 \item[União] é o evento que consiste da união de \textbf{todos} os pontos
   amostrais dos eventos que a compõem. Denotamos a união do evento A
   com B por $A\cup B$.\\
   $A \cup B = \{\omega \in A \text{ ou } \omega \in B\}$
 \item[Interseção] é o evento composto pelos pontos amostrais
   \textbf{comuns} aos eventos que a compõem. Denotamos a interseção de
   A com B por $A\cap B$.\\
   $A \cap B = \{\omega \in A \text{ e } \omega \in B\}$
 \item[Complemento] é o conjunto de pontos do espaço amostral que não
   estão no evento. Denotamos o complemento do evento A por $A^c$.\\
   $A^c = \{\omega \not\in A \}$
\end{description}
\end{block}
\end{frame}

\begin{frame}[fragile]{Tipos de eventos}
\begin{block}{}
\begin{description}
 \item[Disjuntos] (mutuamente exclusivos) são eventos que possuem
   interseção nula, ou seja, $A\cap B = \{\varnothing\}$.
 \item[Complementares] são eventos que a união é o espaço amostral, ou
   seja, $A\cup B = \Omega$.
 %% \item[Exaustivos] (disjuntos e complementares) são eventos que atendem
 %%   ambas propriedades.
\end{description}
\end{block}
\end{frame}

%% \begin{frame}[fragile]{Operações com eventos}
%% \def\firstcircle{(0,0) circle (1cm)}
%% \def\secondcircle{(0:1.5cm) circle (1cm)}

%% \colorlet{circle edge}{green!80}
%% \colorlet{circle area}{green!40}

%% \tikzset{filled/.style={fill=circle area, draw=circle edge, thick},
%%     outline/.style={draw=circle edge, thick}}

%% \begin{center}
%% %\setlength{\parskip}{5mm}
%% % A \cup B
%% \begin{tikzpicture}
%%     \draw[filled] \firstcircle node {$A$}
%%                   \secondcircle node {$B$};
%%     \node[anchor=south] at (current bounding box.north) {$A \cup B$};
%% \end{tikzpicture}
%% % A \cap B
%% \begin{tikzpicture}
%%     \begin{scope}
%%         \clip \firstcircle;
%%         \fill[filled] \secondcircle;
%%     \end{scope}
%%     \draw[outline] \firstcircle node {$A$};
%%     \draw[outline] \secondcircle node {$B$};
%%     \node[anchor=south] at (current bounding box.north) {$A \cap B$};
%% \end{tikzpicture}
%% \end{center}
%% \end{frame}

\begin{frame}[fragile]{Exemplo}
  Considere o lancamento de um dado e os eventos $A = \{1,2,3,4\}$, $B =
  \{\omega:\omega\leq 3\}$, $C = \text{``face par''}$, $D = \text{``face
    primo''}$.
  \begin{itemize}
  \item Uniões
    \begin{itemize}
    \item $A\cup B =$ %\{1,2,3,4\}$
    \item $A\cup C =$ %\{1,2,3,4,6\}$
    \item $A\cup D =$ %\{1,2,3,4,5\}$
    \end{itemize}
  \end{itemize}
  \begin{itemize}
  \item Interseções
    \begin{itemize}
    \item $A\cap B =$ %\{1,2,3\}$
    \item $A\cap C =$ %\{2,4\}$
    \item $A\cap D =$ %\{1,2,3\}$
    \end{itemize}
  \end{itemize}
  \begin{itemize}
  \item Complementos
    \begin{itemize}
    \item $A^c =$ %\{5,6\}$
    \item $B^c =$ %\{\omega: \omega> 3\}$
    \item $D^c =$ %\{4,6\}$
    \end{itemize}
  \end{itemize}
\end{frame}

\begin{frame}[fragile]{Exemplo}
  Considere o lancamento de um dado e os eventos $A = \{1,2,3,4\}$, $B =
  \{\omega:\omega\leq 3\}$, $C = \text{``face par''}$, $D = \text{``face
    primo''}$.
  \begin{itemize}
  \item Uniões
    \begin{itemize}
    \item $A\cup B = \{1,2,3,4\}$
    \item $A\cup C = \{1,2,3,4,6\}$
    \item $A\cup D = \{1,2,3,4,5\}$
    \end{itemize}
  \end{itemize}
  \begin{itemize}
  \item Interseções
    \begin{itemize}
    \item $A\cap B = \{1,2,3\}$
    \item $A\cap C = \{2,4\}$
    \item $A\cap D = \{2,3\}$
    \end{itemize}
  \end{itemize}
  \begin{itemize}
  \item Complementos
    \begin{itemize}
    \item $A^c = \{5,6\}$
    \item $B^c = \{\omega: \omega> 3\}$
    \item $D^c = \{1,4,6\}$
    \end{itemize}
  \end{itemize}
\end{frame}

\begin{frame}[fragile]{Exemplo}
  Sendo $A$ e $B$ dois eventos em um mesmo espaço amostral, ``traduza''
  para a linguagem da Teoria dos Conjuntos as seguintes situações: \\~\\
  \begin{itemize}
  \item[a)] Pelo menos um dos eventos ocorre
  \item[b)] O evento $A$ ocorre, mas $B$ não
  \item[c)] Nenhum deles ocorre
  \end{itemize}
\end{frame}


\section{Probabilidade}

\subsection{Definições de probabilidade}

\begin{frame}{Definições de probabilidade}
  As probabilidades podem ser definidas de diferentes maneiras:
  \begin{itemize}
  \item Definição \textbf{clássica}
  \item Definição \textbf{frequentista}
  \item Definição subjetiva
  \item Definição axiomática
  \end{itemize}
\end{frame}

\begin{frame}{Definições de probabilidade}
  \textbf{Definição Clássica}\\~\\
  Consideramos um espaço amostral $\Omega$ com $n(\Omega)$ eventos simples,
  supondo que sejam \textbf{igualmente prováveis}. Seja $A$ um evento de
  $\Omega$, composto de $n(A)$ eventos simples. A probabilidade de $A$,
  $P(A)$ será
  \begin{equation*}
    P(A) = \frac{n(A)}{n(\Omega)}
  \end{equation*}
\end{frame}

\begin{frame}{Definições de probabilidade}
  Exemplo: Lança-se um dados honesto e observa-se a face voltada para
  cima. Determine a probabilidade de ocorrer a face 4\\~\\ \pause
  \begin{itemize}
  \item Experimento: lançar o dado e observar o resultado da
    face.
  \item Espaço amostral: $\Omega=\{1,2,3,4,5,6\} \quad \Rightarrow \quad
    n(\Omega) = 6$
  \item Pontos amostrais: $\omega_1=\text{1}$, $\omega_2=\text{2}$,
    \ldots, $\omega_6=\text{6}$.
  \item Evento: ocorrer face 4. $A = \{4\} \quad \Rightarrow \quad n(A) = 1$.
  \end{itemize}
  \begin{equation*}
    P(A) = \frac{n(A)}{n(\Omega)} = \frac{1}{6} = 0,1667
  \end{equation*}
\end{frame}

\begin{frame}{Definições de probabilidade}
  Com base nesse resultado, podemos afirmar que a cada 6 lançamentos de
  um dado, uma face será sempre 4? \\~\\ \pause
  \textbf{Não}, pois cada lançamento é aleatório! \\~\\
  No entanto, se repetissemos o lançamento de um dado \textbf{inúmeras
    vezes}, a proporção de vezes em que ocorre o 4 seria aproximadamente
  0,1667 $\Rightarrow$ \textbf{frequência relativa}
\end{frame}

\begin{frame}{Definições de probabilidade}
  \textbf{Definição frequentista}\\~\\
  Podemos então pensar em repetir o experimento aleatório $n$ vezes, e
  contar quantas vezes o evento $A$ ocorre, $n(A)$. Dessa forma a
  frequência relativa de $A$ nas $n$ repetições será
  \begin{equation*}
    f_{n,A} = \frac{n(A)}{n}
  \end{equation*}
  Para $n \rightarrow \infty$ repetições sucessivas e independentes, a
  frequência relativa de $A$ tende para uma constante $p$
  \begin{equation*}
    \lim_{n \rightarrow \infty} \frac{n(A)}{n} = P(A) = p
  \end{equation*}
\end{frame}

\begin{frame}[fragile]{Definições de probabilidade}
  Exemplo: Se um dado fosse lançado \textbf{10} vezes, e contássemos
  quantas vezes saiu a face 4, qual seria a probabilidade desse evento?
\begin{knitrout}\footnotesize
\definecolor{shadecolor}{rgb}{0.969, 0.969, 0.969}\color{fgcolor}\begin{kframe}
\begin{alltt}
\hlcom{## Tamanho da amostra}
\hlstd{n} \hlkwb{<-} \hlnum{10}
\hlcom{## Objeto para armazenar os resultados}
\hlstd{x} \hlkwb{<-} \hlkwd{numeric}\hlstd{(n)}
\hlcom{## Estrutura de repetição}
\hlcom{# Repetir n vezes}
\hlkwa{for}\hlstd{(i} \hlkwa{in} \hlnum{1}\hlopt{:}\hlstd{n)\{}
    \hlcom{# Amostra aleatória de tamanho 1 dos números 1 a 6}
    \hlstd{x[i]} \hlkwb{<-} \hlkwd{sample}\hlstd{(}\hlnum{1}\hlopt{:}\hlnum{6}\hlstd{,} \hlkwc{size} \hlstd{=} \hlnum{1}\hlstd{)}
\hlstd{\}}
\hlcom{## Total de valores igual a 4 => n(A)}
\hlkwd{sum}\hlstd{(x} \hlopt{==} \hlnum{4}\hlstd{)}
\end{alltt}
\begin{verbatim}
[1] 3
\end{verbatim}
\begin{alltt}
\hlcom{## Proporção de valores igual a 4 => n(A)/n}
\hlkwd{sum}\hlstd{(x} \hlopt{==} \hlnum{4}\hlstd{)}\hlopt{/}\hlkwd{length}\hlstd{(x)}
\end{alltt}
\begin{verbatim}
[1] 0.3
\end{verbatim}
\end{kframe}
\end{knitrout}
\end{frame}

\begin{frame}[fragile]{Definições de probabilidade}
  Exemplo: Se um dado fosse lançado \textbf{100} vezes, e contássemos
  quantas vezes saiu a face 4, qual seria a probabilidade desse evento?
\begin{knitrout}\footnotesize
\definecolor{shadecolor}{rgb}{0.969, 0.969, 0.969}\color{fgcolor}\begin{kframe}
\begin{alltt}
\hlcom{## Tamanho da amostra}
\hlstd{n} \hlkwb{<-} \hlnum{100}
\hlcom{## Objeto para armazenar os resultados}
\hlstd{x} \hlkwb{<-} \hlkwd{numeric}\hlstd{(n)}
\hlcom{## Estrutura de repetição}
\hlcom{# Repetir n vezes}
\hlkwa{for}\hlstd{(i} \hlkwa{in} \hlnum{1}\hlopt{:}\hlstd{n)\{}
    \hlcom{# Amostra aleatória de tamanho 1 dos números 1 a 6}
    \hlstd{x[i]} \hlkwb{<-} \hlkwd{sample}\hlstd{(}\hlnum{1}\hlopt{:}\hlnum{6}\hlstd{,} \hlkwc{size} \hlstd{=} \hlnum{1}\hlstd{)}
\hlstd{\}}
\hlcom{## Total de valores igual a 4 => n(A)}
\hlkwd{sum}\hlstd{(x} \hlopt{==} \hlnum{4}\hlstd{)}
\end{alltt}
\begin{verbatim}
[1] 13
\end{verbatim}
\begin{alltt}
\hlcom{## Proporção de valores igual a 4 => n(A)/n}
\hlkwd{sum}\hlstd{(x} \hlopt{==} \hlnum{4}\hlstd{)}\hlopt{/}\hlkwd{length}\hlstd{(x)}
\end{alltt}
\begin{verbatim}
[1] 0.13
\end{verbatim}
\end{kframe}
\end{knitrout}
\end{frame}

\begin{frame}[fragile]{Definições de probabilidade}
  Exemplo: Se um dado fosse lançado \textbf{1000} vezes, e contássemos
  quantas vezes saiu a face 4, qual seria a probabilidade desse evento?
\begin{knitrout}\footnotesize
\definecolor{shadecolor}{rgb}{0.969, 0.969, 0.969}\color{fgcolor}\begin{kframe}
\begin{alltt}
\hlcom{## Tamanho da amostra}
\hlstd{n} \hlkwb{<-} \hlnum{1000}
\hlcom{## Objeto para armazenar os resultados}
\hlstd{x} \hlkwb{<-} \hlkwd{numeric}\hlstd{(n)}
\hlcom{## Estrutura de repetição}
\hlcom{# Repetir n vezes}
\hlkwa{for}\hlstd{(i} \hlkwa{in} \hlnum{1}\hlopt{:}\hlstd{n)\{}
    \hlcom{# Amostra aleatória de tamanho 1 dos números 1 a 6}
    \hlstd{x[i]} \hlkwb{<-} \hlkwd{sample}\hlstd{(}\hlnum{1}\hlopt{:}\hlnum{6}\hlstd{,} \hlkwc{size} \hlstd{=} \hlnum{1}\hlstd{)}
\hlstd{\}}
\hlcom{## Total de valores igual a 4 => n(A)}
\hlkwd{sum}\hlstd{(x} \hlopt{==} \hlnum{4}\hlstd{)}
\end{alltt}
\begin{verbatim}
[1] 146
\end{verbatim}
\begin{alltt}
\hlcom{## Proporção de valores igual a 4 => n(A)/n}
\hlkwd{sum}\hlstd{(x} \hlopt{==} \hlnum{4}\hlstd{)}\hlopt{/}\hlkwd{length}\hlstd{(x)}
\end{alltt}
\begin{verbatim}
[1] 0.146
\end{verbatim}
\end{kframe}
\end{knitrout}
\end{frame}

\begin{frame}[fragile]{Definições de probabilidade}
  Exemplo: Se um dado fosse lançado \textbf{10000} vezes, e contássemos
  quantas vezes saiu a face 4, qual seria a probabilidade desse evento?
\begin{knitrout}\footnotesize
\definecolor{shadecolor}{rgb}{0.969, 0.969, 0.969}\color{fgcolor}\begin{kframe}
\begin{alltt}
\hlcom{## Tamanho da amostra}
\hlstd{n} \hlkwb{<-} \hlnum{10000}
\hlcom{## Objeto para armazenar os resultados}
\hlstd{x} \hlkwb{<-} \hlkwd{numeric}\hlstd{(n)}
\hlcom{## Estrutura de repetição}
\hlcom{# Repetir n vezes}
\hlkwa{for}\hlstd{(i} \hlkwa{in} \hlnum{1}\hlopt{:}\hlstd{n)\{}
    \hlcom{# Amostra aleatória de tamanho 1 dos números 1 a 6}
    \hlstd{x[i]} \hlkwb{<-} \hlkwd{sample}\hlstd{(}\hlnum{1}\hlopt{:}\hlnum{6}\hlstd{,} \hlkwc{size} \hlstd{=} \hlnum{1}\hlstd{)}
\hlstd{\}}
\hlcom{## Total de valores igual a 4 => n(A)}
\hlkwd{sum}\hlstd{(x} \hlopt{==} \hlnum{4}\hlstd{)}
\end{alltt}
\begin{verbatim}
[1] 1586
\end{verbatim}
\begin{alltt}
\hlcom{## Proporção de valores igual a 4 => n(A)/n}
\hlkwd{sum}\hlstd{(x} \hlopt{==} \hlnum{4}\hlstd{)}\hlopt{/}\hlkwd{length}\hlstd{(x)}
\end{alltt}
\begin{verbatim}
[1] 0.1586
\end{verbatim}
\end{kframe}
\end{knitrout}
\end{frame}

\begin{frame}[fragile]{Definições de probabilidade}
  Exemplo: Se um dado fosse lançado \textbf{100000} vezes, e contássemos
  quantas vezes saiu a face 4, qual seria a probabilidade desse evento?
\begin{knitrout}\footnotesize
\definecolor{shadecolor}{rgb}{0.969, 0.969, 0.969}\color{fgcolor}\begin{kframe}
\begin{alltt}
\hlcom{## Tamanho da amostra}
\hlstd{n} \hlkwb{<-} \hlnum{100000}
\hlcom{## Objeto para armazenar os resultados}
\hlstd{x} \hlkwb{<-} \hlkwd{numeric}\hlstd{(n)}
\hlcom{## Estrutura de repetição}
\hlcom{# Repetir n vezes}
\hlkwa{for}\hlstd{(i} \hlkwa{in} \hlnum{1}\hlopt{:}\hlstd{n)\{}
    \hlcom{# Amostra aleatória de tamanho 1 dos números 1 a 6}
    \hlstd{x[i]} \hlkwb{<-} \hlkwd{sample}\hlstd{(}\hlnum{1}\hlopt{:}\hlnum{6}\hlstd{,} \hlkwc{size} \hlstd{=} \hlnum{1}\hlstd{)}
\hlstd{\}}
\hlcom{## Total de valores igual a 4 => n(A)}
\hlkwd{sum}\hlstd{(x} \hlopt{==} \hlnum{4}\hlstd{)}
\end{alltt}
\begin{verbatim}
[1] 16616
\end{verbatim}
\begin{alltt}
\hlcom{## Proporção de valores igual a 4 => n(A)/n}
\hlkwd{sum}\hlstd{(x} \hlopt{==} \hlnum{4}\hlstd{)}\hlopt{/}\hlkwd{length}\hlstd{(x)}
\end{alltt}
\begin{verbatim}
[1] 0.16616
\end{verbatim}
\end{kframe}
\end{knitrout}
\end{frame}

\begin{frame}[fragile]{Definições de probabilidade}
  Exemplo: Se um dado fosse lançado \textbf{1000000} vezes, e contássemos
  quantas vezes saiu a face 4, qual seria a probabilidade desse evento?
\begin{knitrout}\footnotesize
\definecolor{shadecolor}{rgb}{0.969, 0.969, 0.969}\color{fgcolor}\begin{kframe}
\begin{alltt}
\hlcom{## Tamanho da amostra}
\hlstd{n} \hlkwb{<-} \hlnum{1000000}
\hlcom{## Objeto para armazenar os resultados}
\hlstd{x} \hlkwb{<-} \hlkwd{numeric}\hlstd{(n)}
\hlcom{## Estrutura de repetição}
\hlcom{# Repetir n vezes}
\hlkwa{for}\hlstd{(i} \hlkwa{in} \hlnum{1}\hlopt{:}\hlstd{n)\{}
    \hlcom{# Amostra aleatória de tamanho 1 dos números 1 a 6}
    \hlstd{x[i]} \hlkwb{<-} \hlkwd{sample}\hlstd{(}\hlnum{1}\hlopt{:}\hlnum{6}\hlstd{,} \hlkwc{size} \hlstd{=} \hlnum{1}\hlstd{)}
\hlstd{\}}
\hlcom{## Total de valores igual a 4 => n(A)}
\hlkwd{sum}\hlstd{(x} \hlopt{==} \hlnum{4}\hlstd{)}
\end{alltt}
\begin{verbatim}
[1] 166911
\end{verbatim}
\begin{alltt}
\hlcom{## Proporção de valores igual a 4 => n(A)/n}
\hlkwd{sum}\hlstd{(x} \hlopt{==} \hlnum{4}\hlstd{)}\hlopt{/}\hlkwd{length}\hlstd{(x)}
\end{alltt}
\begin{verbatim}
[1] 0.16691
\end{verbatim}
\end{kframe}
\end{knitrout}
\end{frame}

\begin{frame}[fragile]{Definições de probabilidade}
  Assim,
  \begin{equation*}
    \lim_{n \rightarrow \infty} \frac{n(A)}{n} = P(A) \approx 0,1667
  \end{equation*}
  \vspace{1em}

  As probabilidades calculadas a partir de frequências relativas, são
  \textbf{estimativas} da verdadeira probabilidade
  \vspace{1em}
  \begin{block}{Lei dos Grandes Números}
    A Lei dos Grandes Números nos diz que as estimativas dadas pelas
    frequências relativas tendem a ficar melhores com mais observações.
  \end{block}
\end{frame}

\begin{frame}[fragile]{Probabilidades}
  \textbf{Axiomas de probabilidade}\\~\\
  Vamos considerar \textbf{probabilidade} como sendo uma função
  $P(\cdot)$ que associa valores numéricos à um evento $A$ do espaço
  amostral, e que satisfaz as seguintes condições \\~\\
  \begin{itemize}
  \item[i)] $P(\Omega) = 1$; $P(\varnothing) = 0$
  \item[ii)] $0 \leq P(A) \leq 1$ %, \quad \forall A \subset \Omega$
  \item[iii)] $P(A \cup B) = P(A) + P(B) \quad\quad
    \text{se, e seomente se}\quad A \cap B = \varnothing$
  \end{itemize}
  \vspace{1em}
  Os axiomas asseguram que as probabilidades podem ser interpretadas
  como \textbf{frequências relativas}.
\end{frame}

% \begin{frame}[fragile]{Exemplos}
%   Para cada um dos casos abaixo, escreva o espaço amostral $\Omega$ e
%   conte seus elementos \\~\\
%   \begin{itemize}
%   \item[a)] Uma urna contém 5 bolas brancas e 5 bolas azuis. Duas bolas
%     são selecionadas ao acaso, \textbf{com reposição}, e as cores são
%     anotadas.
%   \item[b)] Dois dados são lançados simultaneamente, e estamos
%     interessados na soma das faces observadas.
%   \item[c)] Em uma cidade, famílias com 3 crianças são selecionadas ao
%     acaso, anotando-se o sexo de cada uma delas, de acordo com a idade.
%   \end{itemize}
% \end{frame}

\begin{frame}[fragile]{Exemplos}
  Lançam-se 3 moedas. Determine o espaço amostral. Para cada um dos
  eventos abaixo, descreva os conjuntos e determine as probabilidades: \\~\\
  \begin{itemize}
  \item[a)] Faces iguais
  \item[b)] Cara na 1$^a$ moeda
  \item[c)] Coroa na 2$^a$ e 3$^a$ moedas
  \end{itemize}
\end{frame}

\subsection{Regra da adição}

\begin{frame}[fragile]{Regra da adição}
  \textbf{Notação: tabela de dupla entrada ou tabela de contingência}\\~\\
  \begin{table}
    \centering
    \begin{tabular}{c|cc|c}
      \hline
      & A & B & Total \\
      \hline
      X & $P(A \cap X)$ & $P(B \cap X)$ & $P(X)$ \\
      Y & $P(A \cap Y)$ & $P(B \cap Y)$ & $P(Y)$ \\
      \hline
      Total & $P(A)$ & $P(B)$ & 1 \\
      \hline
    \end{tabular}
  \end{table}
  \begin{itemize}
  \item \textbf{Probabilidades marginais}: são as probabilidades
    individuais nas margens da tabela
  \item \textbf{Probabilidades conjuntas}: são as probabilidades de
    ocorrência de dois eventos simultâneos
  \end{itemize}
\end{frame}

\begin{frame}[fragile]{Regra da adição}
  Considere a tabela de dupla entrada abaixo, que mostra o número de
  estudantes por sexo (F e M) e turma (A e B)
  \begin{table}
    \centering
    \begin{tabular}{c|cc|c}
      \hline
      & F & M & Total \\
      \hline
      A & 21 & 5 & 26 \\
      B & 16 & 8 & 24 \\
      \hline
      Total & 37 & 13 & 50 \\
      \hline
    \end{tabular}
  \end{table}
  Determine a probabilidade de um estudante selecionado ao acaso ser:
  \begin{itemize}
  \item Do sexo feminino, $P(F)$
  \item Do sexo masculino, $P(M)$
  \item Da turma A, $P(A)$
  \item Da turma B, $P(B)$
  \end{itemize}
\end{frame}

\begin{frame}[fragile]{Regra da adição}
  Qual seria a probabilidade de escolhermos um estudante do sexo
  feminino ou da turma B? \\~\\ \pause
  Queremos então $P(F \cup B)$
  \begin{align*}
    P(F \cup B) &= P(F) + P(B) \\
    &= 0,74 + 0,48 \\
    &= 1,22
  \end{align*}
  o que não é possível pois a soma é superior a 1.
\end{frame}

\begin{frame}[fragile]{Regra da adição}
  Não é difícil ver que estamos somando alguns indivíduos 2 vezes, pois:
  \\~\\
  \begin{itemize}
  \item Ao considerarmos apenas estudantes do sexo feminino, temos
    estudantes da turma A bem como da turma B
  \item Ao considerarmos estudantes da turma B, temos estudantes do sexo
    feminino e masculino
  \end{itemize}
  \vspace{1em}
  Assim, os estudantes do sexo feminino e da turma B, ou seja, o evento
  $F \cap B$ está incluído no evento $F$ e no evento $B$ \\~\\
  Logo precisamos subtrair uma vez $P(F \cap B)$ para obter a
  probabilidade correta.
\end{frame}

\begin{frame}[fragile]{Regra da adição}
  Nesse caso, pela tabela, vemos que a interseção $F \cap B$ resulta na
  probabilidade
  \begin{equation*}
    P(F \cap B) = \frac{16}{50} = 0,32
  \end{equation*}
  E o resultado correto para $P(F \cup B)$ é
  \begin{align*}
    P(F \cup B) &= P(F) + P(B) - P(F \cap B) \\
    &= 0,74 + 0,48 - 0,32 \\
    &= 0,9
  \end{align*}
\end{frame}

\begin{frame}[fragile]{Regra da adição}
  \begin{block}{Regra da adição}
    A probabilidade da união entre dois eventos quaisquer, $A$ e $B$, é
    dada pela \textbf{regra da adição de probabilidades}
    \begin{equation*}
      P(A \cup B) = P(A) + P(B) - P(A \cap B)
    \end{equation*}
  \end{block}
  \vspace{1em}
  % Definition of circles
\def\firstcircle{(0,0) circle (1.0cm)}
\def\secondcircle{(0:1.5cm) circle (1.0cm)}

\colorlet{circle edge}{blue!50}
\colorlet{circle area}{blue!20}

\tikzset{filled/.style={fill=circle area, draw=circle edge, thick},
    outline/.style={draw=circle edge, thick}}

%\setlength{\parskip}{5mm}

  \begin{columns}[c]
    \column{.5\textwidth}
    \centering
    \begin{tikzpicture}
      % Set A or B
      \draw[filled] \firstcircle node {$A$}
      \secondcircle node {$B$};
      \node[anchor=south] at (current bounding box.north) {$A \cup B$};
    \end{tikzpicture}
    \column{.5\textwidth}
    \centering
    \begin{tikzpicture}
      % Set A and B
      \begin{scope}
        \clip \firstcircle;
        \fill[filled] \secondcircle;
      \end{scope}
      \draw[outline] \firstcircle node {$A$};
      \draw[outline] \secondcircle node {$B$};
      \node[anchor=south] at (current bounding box.north) {$A \cap B$};
    \end{tikzpicture}
  \end{columns}

  %% % Set A or B but not (A and B) also known a A xor B
  %% \begin{tikzpicture}
  %%   \draw[filled, even odd rule] \firstcircle node {$A$}
  %%   \secondcircle node{$B$};
  %%   \node[anchor=south] at (current bounding box.north) {$\overline{A \cap B}$};
  %% \end{tikzpicture}

%% % Set A but not B
%% \begin{tikzpicture}
%%     \begin{scope}
%%         \clip \firstcircle;
%%         \draw[filled, even odd rule] \firstcircle node {$A$}
%%                                      \secondcircle;
%%     \end{scope}
%%     \draw[outline] \firstcircle
%%                    \secondcircle node {$B$};
%%     \node[anchor=south] at (current bounding box.north) {$A - B$};
%% \end{tikzpicture}

%% % Set B but not A
%% \begin{tikzpicture}
%%     \begin{scope}
%%         \clip \secondcircle;
%%         \draw[filled, even odd rule] \firstcircle
%%                                      \secondcircle node {$B$};
%%     \end{scope}
%%     \draw[outline] \firstcircle node {$A$}
%%                    \secondcircle;
%%     \node[anchor=south] at (current bounding box.north) {$B - A$};
%% \end{tikzpicture}

\end{frame}

\begin{frame}[fragile]{Regra da adição}
  Note que a regra da adição pode ser simplificada, \textbf{se e somente
    se} os eventos $A$ e $B$ forem \textbf{disjuntos} (ou mutuamente
  exclusivos)
    \begin{equation*}
      P(A \cup B) = P(A) + P(B)
    \end{equation*}
    pois, neste caso, $A \cap B = \varnothing \quad \Rightarrow \quad
    P(A \cap B) = P(\varnothing) = 0$
  \vspace{1em}
  % Definition of circles
\def\firstcircle{(0,0) circle (1.0cm)}
\def\secondcircle{(0:2.25cm) circle (1.0cm)}

\colorlet{circle edge}{blue!50}
\colorlet{circle area}{blue!20}

\tikzset{filled/.style={fill=circle area, draw=circle edge, thick},
    outline/.style={draw=circle edge, thick}}

%\setlength{\parskip}{5mm}
  \begin{columns}[c]
    \column{.5\textwidth}
    \centering
    \begin{tikzpicture}
      % Set A or B
      \draw[filled] \firstcircle node {$A$}
      \secondcircle node {$B$};
      \node[anchor=south] at (current bounding box.north) {$A \cup B$};
    \end{tikzpicture}
    \column{.5\textwidth}
    \centering
    \begin{tikzpicture}
      % Set A and B
      \begin{scope}
        \clip \firstcircle;
        \fill[filled] \secondcircle;
      \end{scope}
      \draw[outline] \firstcircle node {$A$};
      \draw[outline] \secondcircle node {$B$};
      \node[anchor=south] at (current bounding box.north) {$A \cap B$};
    \end{tikzpicture}
  \end{columns}
\end{frame}

\begin{frame}[fragile]{Regra da adição}
  \begin{block}{Regra do complementar}
    Como consequência da regra da adição, obtemos que, para qualquer
    evento $A$,
    \begin{equation*}
      P(A) = 1 - P(A^c)
    \end{equation*}
  \end{block}
  \vspace{1em}
  Verifique através de $P(A \cup A^c) = P(A) + P(A^c) - P(A \cap A^c)$
\end{frame}

\begin{frame}[fragile]{Regra da adição}
  Considerando a tabela abaixo, identifique as probabilidades de um
  estudante:
  \begin{table}
    \centering
    \begin{tabular}{c|cc|c}
      \hline
      & F & M & Total \\
      \hline
      A & 21 & 5 & 26 \\
      B & 16 & 8 & 24 \\
      \hline
      Total & 37 & 13 & 50 \\
      \hline
    \end{tabular}
  \end{table}
  \begin{itemize}
  \item[a)] ser do sexo feminino ou masculino
  \item[b)] ser do sexo masculino ou da turma A
  \item[c)] não ser do turma B
  \item[d)] ser da turma A ou da turma B
  \item[e)] não ser do sexo feminino
  \item[f)] ser da turma B ou do sexo masculino
  \end{itemize}
\end{frame}

\subsection{Probabilidade condicional}

\begin{frame}{Probabilidade condicional}
  Em muitas situações práticas, o fenômeno aleatório com o qual
  trabalhamos pode ser separado em etapas.\\~\\
  A informação do que ocorreu em uma determinada etapa pode influenciar
  nas probabilidades de ocorrências das etapas sucessivas.\\~\\
  Nestes casos, dizemos que \textbf{ganhamos informação}, e podemos
  \textsl{recalcular} as probabilidades de interesse. \\~\\
  Estas probabilidades \textsl{recalculadas} recebem o nome de
  \textbf{probabilidade condicional}.
\end{frame}

\begin{frame}{Probabilidade condicional}
 Para entender a ideia de probabilidade condicional, considere o
 seguinte exemplo: \\~\\
 \begin{itemize}
 \item Um dado foi lançado, qual é a probabilidade de ter ocorrido face
   4?
 \item Suponha que o dado foi jogado, e, sem saber o resultado, você
   recebe a informação de que ocorreu face par. Qual é a probabilidade
   de ter saido face 4 com essa ``nova'' informação?
 \end{itemize}
\end{frame}

\begin{frame}{Probabilidade condicional}
$\Omega = \{1,2,3,4,5,6\}$, $n(\Omega) = 6$ \\~\\
$A$ = face 4 = $\{4\}$, $n(A) = 1 \quad \Rightarrow \quad P(A) =
    \frac{n(A)}{n(\Omega)} = \frac{1}{6}$ \\~\\
$B$ = face par = $\{2,4,6\}$, $n(B) = 3 \quad \Rightarrow \quad
P(B) = \frac{n(B)}{n(\Omega)} = \frac{3}{6}$ \\~\\
$C$ = face 4, dado que ocorreu face par = $\{4\}$, $n(C) = 1
\quad \Rightarrow \quad P(C) = \frac{n(C)}{n(B)} = \frac{1}{3}$
\begin{alertblock}{}
  Dado que $B$ tenha ocorrido, o espaço amostral fica \textbf{reduzido}
  para $B$, pois todos os resultados possíveis passam a ser os aqueles
  do evento $B$.
\end{alertblock}
\end{frame}

\begin{frame}{Probabilidade condicional}
  \begin{block}{Definição}
    Para dois eventos $A$ e $B$ de um mesmo espaço amostral, o
    termo $P(A|B)$ denota a probabilidade de $A$ ocorrer, dado que $B$
    ocorreu, e é definido como
    \begin{equation*}
      P(A|B) = \frac{P(A \cap B)}{P(B)}
    \end{equation*}
    da mesma forma que a probabilidade de $B$ ocorrer, dado que $A$
    ocorreu é definida como
    \begin{equation*}
      P(B|A) = \frac{P(B \cap A)}{P(A)}
    \end{equation*}
  \end{block}
\end{frame}

\begin{frame}{Probabilidade condicional}
  Voltando ao exemplo e aplicando a definição de probabilidade
  condicional: \\~\\
  $P(A \cap B) = \frac{n(A \cap B)}{n(\Omega)} = \frac{1}{6}$ \\~\\
  $P(B) = \frac{n(B)}{n(\Omega)} = \frac{3}{6}$
  \vspace{1em}
  \begin{align*}
    P(A|B) &= \frac{P(A \cap B)}{P(B)} \\
     &= \frac{1/6}{3/6} \\
     &= \frac{1}{3}
  \end{align*}
\end{frame}

\begin{frame}{Probabilidade condicional}
  Dessa forma, temos duas maneiras de calcular a probabilidade
  condicional $P(A|B)$: \\~\\
  \begin{enumerate}
  \item Diretamente, pela consideração da probabilidade de $A$ em
    relação ao espaço amostral reduzido $B$
  \item Empregando a definição acima, onde $P(A \cap B)$ e $P(B)$ são
    calculadas em relação ao espaço amostral original $\Omega$
  \end{enumerate}
\end{frame}

\begin{frame}{Probabilidade condicional}
  Considere a tabela abaixo com o número de estudantes por sexo (F e M)
  e turma (A e B):
  \begin{table}
    \centering
    \begin{tabular}{c|cc|c}
      \hline
      & F & M & Total \\
      \hline
      A & 21 & 5 & 26 \\
      B & 16 & 8 & 24 \\
      \hline
      Total & 37 & 13 & 50 \\
      \hline
    \end{tabular}
  \end{table}
  Qual a probabilidade de que um estudante selecionado ao acaso seja da
  turma A, dado que é uma mulher? \\~\\
  Qual a probabilidade de que um estudante selecionado ao acaso seja
  homem, dado que é da turma B?
\end{frame}

\begin{frame}{Exemplo}
  A tabela abaixo fornece um exemplo de 400 itens classificados por
  falhas ($F$ = com falha, $F^c$ = sem falha) na superfície e como
  defeituosos ($D$ = defeituoso, $D^c$ = não defeituoso).
  \begin{table}[!h]
    \centering
    \begin{tabular}{lcc}
      \hline
      \multirow{2}{*}{\textbf{Defeituoso}}
      & \multicolumn{2}{l}{\textbf{Falhas}} \\
      \cline{2-3}
                & $F$         & $F^c$       \\
      \hline
      $D$      & 10           & 18           \\
      $D^c$     & 30           & 342           \\
      \hline
    \end{tabular}
  \end{table}
  Calcule:
  \begin{itemize}
  \item $P(D)$
  \item $P(D|F)$
  \item $P(D|F^c)$
  \end{itemize}
\end{frame}

\subsection{Regra da multiplicação}

\begin{frame}[fragile]{Regra da multiplicação}
  A regra da multiplicação é uma expressão derivada do conceito de
  probabilidade condicional. Uma vez que
  \begin{equation*}
    P(A|B) = \frac{P(A\cap B)}{P(B)}
  \end{equation*}
  temos que
  \begin{equation*}
    P(A\cap B) = P(B) \cdot P(A|B)
  \end{equation*}
  Com isso podemos obter a probabilidade de uma interseção pelo produto
  de uma probabilidade marginal com uma probabilidade condicional.
  \begin{block}{Regra da multiplicação}
    \begin{align*}
      P(A\cap B) &= P(B) \cdot P(A|B) \\
      &= P(A) \cdot P(B|A)
    \end{align*}
  \end{block}
\end{frame}

\begin{frame}[fragile]{Regra da multiplicação} % dantas pg 45
  Essa expressão permite calcular probabilidades em espaços amostrais
  que são realizados em sequência, onde a ocorrência da \textsl{segunda}
  etapa \textbf{depende} da ocorrência da \textsl{primeira} etapa. \\~\\
  \textbf{Exemplo:} Considere uma urna com 3 bolas brancas e 7 bolas
  vermelhas. Duas bolas são retiradas da urna, uma após a outra,
  \textbf{sem reposição}. Determine o espaço amostral e as
  probabilidades associadas a cada ponto amostral.
\end{frame}

\begin{frame}[fragile]{Regra da multiplicação} % dantas pg 45
  Probabilidade de sairem 2 bolas brancas $\{B_1B_2\}$ \\~\\
  %% $P(B_1 \cap B_2) = P(B_2|B_1) \cdot P(B_1) = \frac{3}{10} \frac{2}{9}
  %% = \frac{2}{30}$
  \vspace{1em}
  Probabilidade de sair branca e vermelha $\{B_1V_2\}$ \\~\\
  %% $P(B_1 \cap V_2) = P(V_2|B_1) \cdot P(B_1) = \frac{3}{10} \frac{7}{9}
  %% = \frac{7}{30}$ \\~\\
    \vspace{1em}
  Probabilidade de sair vermelha e branca $\{V_1B_2\}$ \\~\\
  %% $P(V_1 \cap B_2) = P(B_2|V_1) \cdot P(V_1) = \frac{7}{10} \frac{3}{9}
  %% = \frac{7}{30}$ \\~\\
    \vspace{1em}
    Probabilidade de sairem 2 bolas vermelhas $\{V_1V_2\}$ \\~\\
  %% $P(V_1 \cap V_2) = P(V_2|V_1) \cdot P(V_1) = \frac{7}{10} \frac{6}{9}
  %% = \frac{14}{30}$
\end{frame}

\begin{frame}[fragile]{Regra da multiplicação} % dantas pg 45
  Probabilidade de sairem 2 bolas brancas $\{B_1B_2\}$ \\~\\
  $P(B_1 \cap B_2) = P(B_1)P(B_2|B_1)  = \frac{3}{10} \frac{2}{9}
  = \frac{2}{30}$ \\~\\
  Probabilidade de sair branca e vermelha $\{B_1V_2\}$ \\~\\
  $P(B_1 \cap V_2) = P(B_1)P(V_2|B_1)  = \frac{3}{10} \frac{7}{9}
  = \frac{7}{30}$ \\~\\
  Probabilidade de sair vermelha e branca $\{V_1B_2\}$ \\~\\
  $P(V_1 \cap B_2) = P(V_1)P(B_2|V_1)  = \frac{7}{10} \frac{3}{9}
  = \frac{7}{30}$ \\~\\
    Probabilidade de sairem 2 bolas vermelhas $\{V_1V_2\}$ \\~\\
  $P(V_1 \cap V_2) = P(V_1)P(V_2|V_1)  = \frac{7}{10} \frac{6}{9}
  = \frac{14}{30}$
\end{frame}

%% Aqui eu resolvi dis exercicios do livro da Suzi. O 16 mostra a
%% extensao da regra da multiplicacao para 3 eventos. O 8 mostra a ideia
%% de independencia de eventos (começe pela letra b e depois faça a
%% a). Este ultimo mostra que a probabilidade nao muda quando tem
%% reposicao e portanto os eventos sao independentes. Ja da o gancho
%% para o proximo topico...

\begin{frame}{Exemplo}
  \begin{enumerate}
  \item Num lote de 12 peças, 4 são defeituosas. Três peças são
    retiradas aleatoriamente, uma após a outra, sem reposição. Encontre
    a probabilidade de todas essas três peças serem não-defeituosas.
  \item Qual a probabilidade de se obter dois ases em seguida, quando se
    extraem duas cartas de um baralho comum de 52 cartas, se:
    \begin{enumerate}
    \item A primeira carta extraída não é reposta antes da extração da
      segunda carta.
    \item A primeira carta é reposta no baralho antes da extração da
      segunda carta.
    \end{enumerate}
  \end{enumerate}
\end{frame}

\subsection{Independência de eventos}

\begin{frame}{Independência de eventos}
  Vimos que para probabilidades condicionais, $P(A|B)$, saber que $B$
  ocorreu nos dá uma informação ``extra'' sobre a ocorrência de $A$ \\~\\
  Porém, existem algumas situações nas quais saber que o evento $B$
  ocorreu, não tem qualquer interferência na ocorrência ou não de $A$\\~\\
  Nestes casos, podemos dizer que os aventos $A$ e $B$ são
  \textbf{independentes}
\end{frame}

\begin{frame}{Independência de eventos}
  Os eventos A e B são \textbf{eventos independentes} se a ocorrência de
  B não altera a probabilidade de ocorrência de A, ou seja, eventos A e
  B são independentes se
  \begin{equation*}
    P(A|B) = P(A) \quad \text{e também que} \quad P(B|A) = P(B)
  \end{equation*}
  Com isso, e a regra da multiplicação, temos que
  \begin{block}{}
    \begin{align*}
      P(A \cap B) &= P(B) \cdot P(A|B) = P(B) \cdot P(A) \\
      P(A \cap B) &= P(A) \cdot P(B|A) = P(A) \cdot P(B)
    \end{align*}
  \end{block}
  Isso significa que se dois eventos são independentes, a probabilidade
  de ocorrência simultânea $P(A \cap B)$ é o produto das probabilidades
  marginais, $P(A)$ e $P(B)$.
\end{frame}

\begin{frame}{Independência de eventos}
  Dessa forma, podemos verificar se dois eventos são independentes de
  duas formas: \\~\\
  \begin{enumerate}
  \item Pela definição intuitiva
    \begin{align*}
      P(A|B) &= P(A) \\
      P(B|A) &= P(B)
    \end{align*}
    \textbf{Observação:} se o evento A é independente do evento B, então
    nós esperamos que B também seja independente de A.
    %% \textbf{Importante:} as duas condições precisam ser satisfeitas ao
    %% mesmo tempo para que $A$ e $B$ sejam independentes (se apenas uma
    %% for satisfeita, os eventos não são independentes)
  \item Pela definição formal
    \begin{align*}
      P(A \cap B) = P(A) \cdot P(B)
    \end{align*}
  \end{enumerate}
\end{frame}

\begin{frame}[fragile]{Exemplo}
  \begin{block}{Lançamento de um dado}
    Considere o lançamento de um dado e os seguintes eventos
    \begin{center}
      $A = \text{``resultado é um número par''}$\\
      $B = \text{``resultado é um número menor ou igual a 4''}$
    \end{center}
    Os eventos $A$ e $B$ são independentes?
  \end{block}
  \begin{center}
    \begin{tikzpicture}[scale=0.5]
      % \draw[help lines] (0,0) grid (8,6);
      \draw[thick] (0,0) rectangle +(8,6);
      \fill[blue!20, draw=black, thick] (5,3) circle (1.75cm);
      \draw[thick] (3,3) circle (1.75cm);
      \node (p1) at (5,5) {$B$};  \node (p2) at (3,1) {$A$};
      \fill[red] (5.5,3.5) circle (3pt) node [above, color=black] {$1$};
      \fill[red] (5.5,2) circle (3pt) node [above, color=black] {$3$};
      \fill[red] (2.5,3) circle (3pt) node [above, color=black] {$6$};
      \fill[red] (4,2.2) circle (3pt) node [above, color=black] {$4$};
      \fill[red] (4,3.5) circle (3pt) node [above, color=black] {$2$};
      \fill[red] (3,5.2) circle (3pt) node [above, color=black] {$5$};
    \end{tikzpicture}
  \end{center}
\end{frame}

\begin{frame}{Exemplo}
  \textbf{Pela definição intuitiva}: \\~\\
  $P(A) = 1/2$, \quad $P(A|B) = P(A\cap B)/P(B) = \frac{2/6}{4/6} =
  1/2$ \\~\\
  $P(B) = 2/3$, \quad $P(B|A) = P(B\cap A)/P(A) = \frac{2/6}{3/6} =
  2/3$. \\~\\
  Portanto: $P(A|B) = P(A)$ e $P(B|A) = P(B)$ \\~\\
  \textbf{Pela definição formal:} \\~\\
  $P(A \cap B) = P(A)P(B) = \frac{1}{2} \cdot \frac{2}{3} = 1/3$ \\~\\
  $P(A \cap B) = \frac{2}{6} = 1/3$, assim $P(A \cap B) = P(A)P(B)$ \\~\\
  Portanto, os eventos $A$ e $B$ são independentes. Saber que $A$
  ocorreu não muda a probabilidade de $B$ ocorrer e vice-versa.
\end{frame}

\begin{frame}{Exemplo}
  \begin{enumerate}
  \item As probabilidades de um estudante ser aprovado em exames de
    matemática, inglês, ou de ambos são
    \begin{equation*}
      P(M) = 0,7 \qquad P(I) = 0,8 \qquad P(M \cap I) = 0,56
    \end{equation*}
    Verifique se os eventos M e I são independentes.
  \item As probabilidades de chover em determinada cidade nos
    dias de natal (N), no dia de ano-novo (A), ou em ambos os dias são
    \begin{equation*}
      P(N) = 0,6 \qquad P(A) = 0,6 \qquad P(N \cap A) = 0,42
    \end{equation*}
    Verifique se os eventos N (``chover no natal'') e A (``chover no ano
    novo'') são independentes.
  \end{enumerate}
\end{frame}

\begin{frame}{Exemplo}
  A tabela abaixo fornece um exemplo de 400 itens classificados por
  falhas ($F$ = com falha, $F^c$ = sem falha) na superfície e como
  defeituosos ($D$ = defeituoso, $D^c$ = não defeituoso).
  \begin{table}[!h]
    \centering
    \begin{tabular}{lcc}
      \hline
      \multirow{2}{*}{\textbf{Defeituoso}}
      & \multicolumn{2}{l}{\textbf{Falhas}} \\
      \cline{2-3}
                & $F$         & $F^c$       \\
      \hline
      $D$      & 2           & 18           \\
      $D^c$     & 38           & 342           \\
      \hline
    \end{tabular}
  \end{table}
  Calcule:
  \begin{itemize}
  \item $P(D)$
  \item $P(D|F)$
  \item Compare estes resultados com aqueles do slide 51, e verifique,
    em cada caso, se os eventos $D$ e $F$ são independentes.
  \end{itemize}
\end{frame}

% \subsection{Exercícios}

% \begin{frame}[fragile]{Exercícios}
%   Exercícios 1--20, do capítulo 5 do livro (pgs. 135--141).
%   \vspace{1em}
%   \begin{itemize}
%   \item[] Pinto, SS; Silva, CS. \textbf{Estatística, Vol I}. Rio Grande:
%     Editora da FURG, 2010. [Cap. 5]
%   \end{itemize}
% \end{frame}

\section{Referências}


\begin{frame}{Referências}
  \begin{itemize}
  \item Bussab, WO; Morettin, PA. \textbf{Estatística básica}. São
    Paulo: Saraiva, 2006. [Cap. 5]
  \item Magalhães, MN; Lima, ACP. \textbf{Noções de Probabilidade e
      Estatística}. São Paulo: EDUSP, 2008. [Cap. 2]
  \item Montgomery, DC; Runger, GC. \textbf{Estatística aplicada e
      probabilidade para engenheiros}. Rio de Janeiro: LTC Editora,
    2012. [Cap. 2]
  \end{itemize}
\end{frame}

\end{document}

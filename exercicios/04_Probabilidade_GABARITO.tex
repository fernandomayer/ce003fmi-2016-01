\documentclass[a4paper,11pt,fleqn]{article}\usepackage[]{graphicx}\usepackage[]{color}
%% maxwidth is the original width if it is less than linewidth
%% otherwise use linewidth (to make sure the graphics do not exceed the margin)
\makeatletter
\def\maxwidth{ %
  \ifdim\Gin@nat@width>\linewidth
    \linewidth
  \else
    \Gin@nat@width
  \fi
}
\makeatother

\definecolor{fgcolor}{rgb}{0, 0, 0}
\newcommand{\hlnum}[1]{\textcolor[rgb]{0,0,0}{#1}}%
\newcommand{\hlstr}[1]{\textcolor[rgb]{0,0,0}{#1}}%
\newcommand{\hlcom}[1]{\textcolor[rgb]{0.4,0.4,0.4}{\textit{#1}}}%
\newcommand{\hlopt}[1]{\textcolor[rgb]{0,0,0}{\textbf{#1}}}%
\newcommand{\hlstd}[1]{\textcolor[rgb]{0,0,0}{#1}}%
\newcommand{\hlkwa}[1]{\textcolor[rgb]{0,0,0}{\textbf{#1}}}%
\newcommand{\hlkwb}[1]{\textcolor[rgb]{0,0,0}{\textbf{#1}}}%
\newcommand{\hlkwc}[1]{\textcolor[rgb]{0,0,0}{\textbf{#1}}}%
\newcommand{\hlkwd}[1]{\textcolor[rgb]{0,0,0}{\textbf{#1}}}%

\usepackage{framed}
\makeatletter
\newenvironment{kframe}{%
 \def\at@end@of@kframe{}%
 \ifinner\ifhmode%
  \def\at@end@of@kframe{\end{minipage}}%
  \begin{minipage}{\columnwidth}%
 \fi\fi%
 \def\FrameCommand##1{\hskip\@totalleftmargin \hskip-\fboxsep
 \colorbox{shadecolor}{##1}\hskip-\fboxsep
     % There is no \\@totalrightmargin, so:
     \hskip-\linewidth \hskip-\@totalleftmargin \hskip\columnwidth}%
 \MakeFramed {\advance\hsize-\width
   \@totalleftmargin\z@ \linewidth\hsize
   \@setminipage}}%
 {\par\unskip\endMakeFramed%
 \at@end@of@kframe}
\makeatother

\definecolor{shadecolor}{rgb}{.97, .97, .97}
\definecolor{messagecolor}{rgb}{0, 0, 0}
\definecolor{warningcolor}{rgb}{1, 0, 1}
\definecolor{errorcolor}{rgb}{1, 0, 0}
\newenvironment{knitrout}{}{} % an empty environment to be redefined in TeX

\usepackage{alltt}

%%----------------------------------------------------------------------
%% opções comuns
\usepackage[brazilian]{babel}
\usepackage[utf8]{inputenc}
\usepackage[T1]{fontenc}
\usepackage{textcomp}
%\usepackage[margin=2cm]{geometry}
\usepackage{indentfirst}
\usepackage{fancybox}
%\usepackage[usenames,dvipsnames]{color}
\usepackage{amsmath,amsfonts,amssymb,amsthm}
\usepackage{lscape}
\usepackage{natbib}
\setlength{\bibsep}{0.0pt}
\usepackage{url}
\usepackage{multicol}
\usepackage{multirow}
\usepackage[final]{pdfpages}
\usepackage{setspace}
\usepackage{paralist} % enumitem, compactitem
  \plitemsep=2pt
%%----------------------------------------------------------------------

%%----------------------------------------------------------------------
%% FLOATS: graficos e tabelas
\usepackage{graphicx}
\usepackage{float} % fornece a opção [H] para floats
\usepackage{longtable}
\usepackage{supertabular}
%% captions e headings em sans-serif
\usepackage[font={sf},labelfont={sf,bf}]{caption}
\usepackage{subcaption}
\renewcommand{\thesubfigure}{\Alph{subfigure}}
\usepackage{titlesec}
\titleformat*{\section}{\normalsize\bfseries\sffamily}
\titleformat*{\subsection}{\normalsize\bfseries\sffamily}
\titleformat*{\subsubsection}{\normalsize\bfseries\sffamily}
\titleformat*{\paragraph}{\normalsize\bfseries\sffamily}
\titleformat*{\subparagraph}{\normalsize\bfseries\sffamily}
\theoremstyle{definition}
\newtheorem*{mydef}{Definição}
%%----------------------------------------------------------------------

%%----------------------------------------------------------------------
%% definiçoes de hyperref e xcolor
\usepackage{hyperref}
\usepackage{xcolor}
%%----------------------------------------------------------------------

%%----------------------------------------------------------------------
%% FONTES

%% micro-tipografia
\usepackage[protrusion=true,expansion=true]{microtype}
%% Bitstream Charter with mathdesign
\usepackage{lmodern} % sans-serif: Latin Modern
\usepackage[charter]{mathdesign} % serif: Bitstream Charter
\usepackage[scaled]{beramono} % truetype: Bistream Vera Sans Mono
\usepackage[scaled]{helvet}
%\usepackage{inconsolata}


%\usepackage[sf]{titlesec}
%%----------------------------------------------------------------------

%%----------------------------------------------------------------------
%% hifenização
\usepackage[htt]{hyphenat} % permite hifenizar texttt. Ao inves disso
% pode usar \allowbreak no ponto qu quiser quebrar dentro do texttt
\hyphenation{con-si-de-ra-ção pes-que-i-ros pes-que-i-ra se-gui-do-ras
  di-fe-ren-tes pla-ni-lha pla-ni-lhão re-fe-ren-te con-ta-gem
  em-bar-ques qua-li-da-de a-le-a-to-ri-za-dos}
%%----------------------------------------------------------------------

%%----------------------------------------------------------------------
%% comandos customizados
\usepackage{xspace} % lida com os espaços depois dos comandos
\providecommand{\eg}{\textit{e.g.}\xspace}
\providecommand{\ie}{\textit{i.e.}\xspace}
\providecommand{\R}{\textsf{R}\xspace}
\newcommand{\mb}[1]{\mathbf{#1}}
\newcommand{\bs}[1]{\boldsymbol{#1}}
\providecommand{\E}{\text{E}}
\providecommand{\Var}{\text{Var}}
\providecommand{\logit}{\text{logit}}
%% Para alterar o titulo do thebibliography
\addto\captionsbrazilian{%
  \renewcommand{\refname}{Bibliografia}
}
%%----------------------------------------------------------------------

%%----------------------------------------------------------------------
%% Comandos para deixar o texto mais compacto
\usepackage{marginnote}
\usepackage[top=1cm, bottom=1cm, inner=1cm, outer=1cm,nohead, nofoot, heightrounded, marginparsep=.05cm]{geometry}
\setlength{\parindent}{0pt}
%%----------------------------------------------------------------------
\IfFileExists{upquote.sty}{\usepackage{upquote}}{}
\begin{document}

\reversemarginpar % para colocar a marginnote a esquerda





\hrule
\vspace{0.3cm}

\begin{minipage}[c]{.85\textwidth}
  Estatística II --- CE003 \\
  Prof. Fernando de Pol Mayer --- Departamento de Estatística --- DEST \\
  Exercícios: probabilidade \\
  Nome: GABARITO   \hfill GRR: \hspace{2cm}
\end{minipage}\hfill
\begin{minipage}[c]{.15\textwidth}
\flushright
\includegraphics[width=2.2cm]{../img/ufpr-logo.png}
\end{minipage}

\vspace{0.3cm}
\hrule
\vspace{0.3cm}
%%----------------------------------------------------------------------

\begin{compactenum} % Magalhaes, pg 40
\item Para cada um dos eventos abaixo, escreva o espaço amostral
  correspondente e conte seus elementos:
  \begin{compactenum}
  \item $\Omega = \{CC, CR, RC, RR\} \quad n(\Omega) = 4$
  \item $\Omega = \{PP, PI, IP, II\} \quad n(\Omega) = 4$
  \item $\Omega = \{AA, AV, VA, VV\} \quad n(\Omega) = 4$
  \item $\Omega = \{2, 3, 4, \ldots, 12\} \quad n(\Omega) = 11$
  \item $\Omega = \{MMM, MMF, MFM, FMM, FFM, FMF, MFF, FFF\} \quad
  n(\Omega) = 8$
  \item $\Omega = \{\omega : 0 \leq \omega \leq 20 \} \quad n(\Omega) =
  21$
  \item $\Omega = \{C, RC, RRC, RRRC, RRRRC, \ldots\} \quad n(\Omega) =
    \infty$
    % A partir daqui: Bussab, pg 108, cap 5
  \item $\Omega = \{\omega : \omega > 0 \} = \mathbb{R}^{+} \quad
    n(\Omega) = \infty$
  \item $\Omega = \{3, 4, 5, \ldots, 10\} \quad n(\Omega) = 8$
  \item $\Omega = \{1, 2, 3, \ldots\} \quad n(\Omega) = \infty$
  \item $\Omega = \{AA, AB, AC, AD, AE, BA, BB, BC, BD, BE, CA, CB, CC,
    CD, CE, DA, DB, DC, DD, DE, \\ EA, EB, EC, ED, EE\} \quad n(\Omega) =
    25$
  \item $\Omega = \{AB, AC, AD, AE, BA, BC, BD, BE, CA, CB,
    CD, CE, DA, DB, DC, DE, EA, EB, EC, ED\} \quad n(\Omega) =
    20$
  \item $\Omega = \{AB, AC, AD, AE, BC, BD, BE,
    CD, CE, DE\} \quad n(\Omega) = 10$
  \end{compactenum}

\vspace{0.3cm}
\hrule
\vspace{0.3cm}

% Bussab, pg 107, cap. 5
\item $\Omega = \{BC, BR, VB, VV\}$

\vspace{0.3cm}
\hrule
\vspace{0.3cm}

% Ross, pg 71
\item
  \begin{compactenum}
  \item $\Omega = \{VV, VA, VB, AA, AV, AB, BB, BA, BV\}$
  \item $\Omega = \{VA, VB, AV, AB, BA, BV\}$
  \end{compactenum}

\vspace{0.3cm}
\hrule
\vspace{0.3cm}

% Montgomery, pg 19, cap 2
\item
  \begin{compactenum}
  \item $\Omega = \{x : x > 0\}$ \,
  \item $A \cup B = \{x : x > 11 \}$ \,
  \item $A \cap B = \{x : 11 < x \leq 15\}$ \,
  \item $A^{c} = \{x : x \leq 11 \}$ \,
  \item $A \cup B \cup C = \{x : x \geq 8\}$ \,
  \item $(A \cup C)^{c} = \{x : x < 8\}$ \,
  \item $A \cap B \cap C = \varnothing$ \,
  \item $B^{c} \cap C = \varnothing$ \,
  \item $A \cup (B \cap C) = \{x : x \geq 8\}$
  \end{compactenum}

\vspace{0.3cm}
\hrule
\vspace{0.3cm}

% Montgomery, pg 19, cap 2
\item $\Omega = \{\omega : \omega \geq 0\}$
  \begin{compactenum}
  \item $A = \{\omega : 675 \leq \omega \leq 700 \}$
  \item $B = \{\omega : 450 \leq \omega \leq 500 \}$ \,
  \item $A \cap B = \varnothing$ \,
  \item $A \cup B = \{\omega : 450 \leq \omega \leq 500 \cup 675 \leq
    \omega \leq 700\}$
  \end{compactenum}

\vspace{0.3cm}
\hrule
\vspace{0.3cm}

% Montgomery, pg 19, cap 2
\item $\Omega = \{PPP, PPN, PNP, NPP, PNN, NPN, NNP, NNN\}$
  \begin{compactenum}
  \item $A = \{PPP\}$
  \item $B = \{NNN\}$
  \item $A \cap B = \varnothing$ \,
  \item $A \cup B = \{PPP, NNN\}$
  \end{compactenum}

\vspace{0.3cm}
\hrule
\vspace{0.3cm}

\clearpage

\vspace{0.3cm}
\hrule
\vspace{0.3cm}

% Bussab, pg 113, cap. 5
\item
  \begin{compactenum}
  \item $A = \{(3,6), (4,5), (5,4), (6,3)\}$ e $B = \{(4,1), (4,2),
    \ldots (6,6)\}$
  \item $A \cup B = \{(3,6), (4,1), (4,2), \ldots (6,6)$,
    $A \cap B = \{(4,5), (5,4), (6,3)\}$, e \\ $A^{c} = \{(1,1), (1,2),
    \ldots (3,5), (4,1), \ldots, (4,4), (4,6), (5,1), \ldots, (5,3),
    (5,5), (5,6), \ldots, (6,2), (6,4), (6,5), (6.6)\}$.
  \item $P(A) = 0,11$, $P(B) = 0,5$, $P(A \cup B) = 0,53$, $P(A \cap B)
    = 0,083$, $P(A^{c}) = 0,89$
  \end{compactenum}

\vspace{0.3cm}
\hrule
\vspace{0.3cm}

% Bussab, pg 126, cap. 5
\item
  \begin{inparaenum}
  \item $0,0296$
  \item $0,0298$
  \end{inparaenum}

\vspace{0.3cm}
\hrule
\vspace{0.3cm}

% Bussab, pg 126, cap. 5
\item
  \begin{inparaenum}
  \item 0,049
  \item 0,463
  \item 0,295
  \end{inparaenum}

\vspace{0.3cm}
\hrule
\vspace{0.3cm}

% Ross pg 71
\item
  \begin{inparaenum}
  \item $0,8$
  \item $0,3$
  \item $0$
  \end{inparaenum}

\vspace{0.3cm}
\hrule
\vspace{0.3cm}

% Montgomery, pg. 26, cap 2
\item
    \begin{inparaenum}
    \item $0,3$
    \item $0,4$
    \item $0,1$
    \item $0,2$
    \item $0,6$
    \item $0,8$
    \end{inparaenum}

\vspace{0.3cm}
\hrule
\vspace{0.3cm}

% Montgomery, pg. 27, cap 2
\item
    \begin{inparaenum}
    \item $0,9$
    \item $0$
    \item $0$
    \item $0$
    \item $0,1$
    \end{inparaenum}

\vspace{0.3cm}
\hrule
\vspace{0.3cm}

% Montgomery, pg 19, cap 2
\item
  \begin{compactenum}
  \item $A \cap B = 70$, $A^{c} = 14$, e $A \cup B = 95$.
  \item
    \begin{inparaenum}
    \item $0,86$ \,
    \item $0,79$ \,
    \item $0,14$ \,
    \item $0,7$ \,
    \item $0,95$ \,
    \item $0,84$ \,
    \item $70/79$ \,
    \item $70/86$
    \end{inparaenum}
  \item $0,7$
  \item $0,95$
  \item Não, pois $P(A \cap B) = 70/100$
  \item Não, pois $P(A \cap B) \neq P(A)P(B)$
  \end{compactenum}

\vspace{0.3cm}
\hrule
\vspace{0.3cm}

% magalhaes pg 41
\item $0,4$

\vspace{0.3cm}
\hrule
\vspace{0.3cm}

% Montgomery, pg 31, cap 2
\item
  \begin{inparaenum}
  \item $20/100$
  \item $19/99$
  \item $0,038$
  \item $0,2$
  \end{inparaenum}

\vspace{0.3cm}
\hrule
\vspace{0.3cm}

% Montgomery, pg 21, cap 2
\item
  \begin{compactenum}
  \item
    \begin{inparaenum}
    \item $36$ \,
    \item $92$ \,
    \item $148$ \,
    \item $168$ \,
    \item $56$
    \end{inparaenum}
  \item
    \begin{inparaenum}
    \item $0,2745$ \,
    \item $0,4510$ \,
    \item $0,7255$ \,
    \item $0,8235$ \,
    \item $0,2745$ \,
    \item $0,7255$ \,
    \item $0,6087$ \,
    \item $0,3913$ \,
    \item $0,5$ \,
    \item $0,5$
    \end{inparaenum}
  \item Como $P(A) P(B) = (0.5490)(0.4510) = 0.2476 \neq 0.2745 = P(A \cap
    B)$, não são independentes.
  \end{compactenum}

\vspace{0.3cm}
\hrule
\vspace{0.3cm}

% Montgomery pg 33, cap 2
\item
  \begin{inparaenum}
  \item $0,2$ \,
  \item $0,3$
  \end{inparaenum}

\vspace{0.3cm}
\hrule
\vspace{0.3cm}

% Montgomery pg 33, cap 2
\item $0,22$

\vspace{0.3cm}
\hrule
\vspace{0.3cm}


% Montgomery, pg 21, cap 2
\item
  \begin{compactenum}
  \item
    \begin{inparaenum}
    \item $673$ \,
    \item $1672$ \,
    \item $6915$ \,
    \item $8399$ \,
    \item $1578$
    \end{inparaenum}
  \item
    \begin{inparaenum}
    \item $0,0792$ \,
    \item $0,1969$ \,
    \item $0,8142$ \,
    \item $0,9889$ \,
    \item $0,1858$ \,
    \item $0,9208$ \,
    \item $0,877$
    \end{inparaenum}
  \item $0,0987$
  \item $0,0650$
  \item Como $P(A)P(B) = (0.8031)(0.0903) = 0.0725 \neq 0.0792 = P(A
    \cap B)$, não são independentes.
  \end{compactenum}

\vspace{0.3cm}
\hrule
\vspace{0.3cm}

% Montgomery, pg 20, cap 2
\item
  \begin{inparaenum}
  \item $A^{c} = \{x : x > 72,5\}$ \,
  \item $B^{c} = \{x : x \leq 52,5\}$ \,
  \item $A \cap B = \{x : 52,5 < x \leq 72,5\}$ \,
  \item $A \cup B = \{x : x > 0\}$
  \end{inparaenum}

\vspace{0.3cm}
\hrule
\vspace{0.3cm}

% Montgomery, pg 33, cap 2
\item
  \begin{inparaenum}
  \item $0,0949$
  \item $0,3828$
  \end{inparaenum}

\vspace{0.3cm}
\hrule
\vspace{0.3cm}

% Montgomery, pg 20, cap 2
\item
  \begin{compactenum}
  \item $\{ab, ac, ad, bc, bd, cd, ba, ca, da, cb, db, dc\}$
  \item $\{ab, ac, ad, ae, af, ag, bc, bd, be, bf, bg, cd, ce, cf, cg,
    ef, eg, fg, ba, ca, da, ea, fa, ga, cb, db, eb, \\ fb, gb, dc, ec, fc,
    gc, fe, ge, gf\}$
  \item $\Omega = \{BB, BD, DB, DD\}$
  \item $\Omega = \{BB, BD, DB\}$
  \end{compactenum}

\vspace{0.3cm}
\hrule
\vspace{0.3cm}

\clearpage

\vspace{0.3cm}
\hrule
\vspace{0.3cm}

% Montgomery, pg 22, cap 2
\item
  \begin{inparaenum}
  \item $2/5$ \,
  \item $3/5$ \,
  \item $3/5$ \,
  \item $1$ \,
  \item $0$
  \end{inparaenum}

% Montgomery, pg 22, cap 2
\item
  \begin{inparaenum}
  \item $0,4$ \,
  \item $0,8$ \,
  \item $0,6$ \,
  \item $1$ \,
  \item $0,2$
  \end{inparaenum}

\vspace{0.3cm}
\hrule
\vspace{0.3cm}


% Montgomery, pg 20, cap 2
\item
  \begin{compactenum}
  \item $\Omega = \{BB, BP, BG, PB, PP, PG, GB, GP, GG\}$
  \item $\Omega = \{BB, BP, BG, PB, PP, PG, GB, GP\}$
  \end{compactenum}

\vspace{0.3cm}
\hrule
\vspace{0.3cm}

% Montgomery, pg. 24, cap. 2
\item $\Omega = \{95, 96, \ldots, 104\}$
  \begin{inparaenum}
  \item $0,1$
  \item $0,5$
  \item $0,5$
  \item
    \begin{inparaenum}
    \item $0,01$
    \item $0,49$
    \end{inparaenum}
  \end{inparaenum}

\vspace{0.3cm}
\hrule
\vspace{0.3cm}

% Ross pg 71
\item $0,74$

\vspace{0.3cm}
\hrule
\vspace{0.3cm}

% Montgomery, pg. 24, cap. 2
\item
  \begin{inparaenum}
  \item $0,83$
  \item $0,85$
  \end{inparaenum}

\vspace{0.3cm}
\hrule
\vspace{0.3cm}

% Magalhaes, pg 41
\item Tabela de contingência \\
  \begin{table}[h]
    \centering
    \begin{tabular}{lcccc}
      \hline
      & \textbf{Economia} (E) & \textbf{Administração} (A) & \textbf{Outros} (O) &
      \textbf{Total} \\
      \hline
      \textbf{Esportista} (Es) & 100 & 200 & 3700 & 4000 \\
      \textbf{Não esportista} (Es$^c$) & 400 & 500 & 5100 & 6000 \\
      \hline
      \textbf{Total} & 500 & 700 & 8800 & 10000 \\
      \hline
    \end{tabular}
  \end{table}

  \begin{inparaenum}
  \item 4000/10000 = 2/5
  \item 200/10000 = 1/50
  \item 8800/10000 = 22/25
  \item 5100/10000 = 51/100
  \item 100/500 = 1/5
  \item 500/6000 = 1/12
  \item 5100/8800 = 51/88
  \item 3700/4000 = 37/40
  \end{inparaenum}

\vspace{0.3cm}
\hrule
\vspace{0.3cm}

% Kazmier, pg 78
\item
  \begin{compactenum}
  \item Sim, porque não é possível receber conceito A e B ao mesmo
    tempo.
  \item 0
  \item $P(A \cap B) = 0 \neq 0,24 = P(A) \cdot P(B)$. São dependentes.
  \end{compactenum}

\vspace{0.3cm}
\hrule
\vspace{0.3cm}

% morettin, pg 23
\item
  \begin{inparaenum}
  \item $0,4$
  \item $0,5$
  \end{inparaenum}

\vspace{0.3cm}
\hrule
\vspace{0.3cm}

% morettin, pg 40
\item
  \begin{inparaenum}
  \item $3/8$
  \item $1/2$
  \item $1/2$
  \end{inparaenum}

\vspace{0.3cm}
\hrule
\vspace{0.3cm}

% kazmier, pg 77
\item
  \begin{inparaenum}
  \item
  \item $0,5$
  \item $0,3$
  \item $0,5$
  \item $P(C|E) = 0,5 \neq 0,3 = P(C)$. São dependentes.
  \end{inparaenum}

\vspace{0.3cm}
\hrule
\vspace{0.3cm}

% kazmier, pg 78
\item $0,064$

\vspace{0.3cm}
\hrule
\vspace{0.3cm}

% Montgomery, pg 31, cap 2
\item
  \begin{compactenum}
  \item $0,0736$
  \item $0,1013$
  \item Calcular todas as probabilidades condicionais de um nível para
    outro. Fase madura da larva.
  \end{compactenum}

\vspace{0.3cm}
\hrule
\vspace{0.3cm}

% Montgomery, pg 36, cap 2
\item Não, pois $P(A|B) \neq P(A)$.

\vspace{0.3cm}
\hrule
\vspace{0.3cm}

% Montgomery, pg 36, cap 2
\item $P(A^{c}) = 1 - P(A) = 0.7$ e $P(A^{c}|B) = 1 - P(A|B) = 0.7$,
  portanto são independentes.

\vspace{0.3cm}
\hrule
\vspace{0.3cm}

% Montgomery, pg 36, cap 2
\item Como $A$ e $B$ são mutuamente excludentes, então $P(A \cap B) = 0$
  e $P(A)P(B) = 0.04$, portanto não são independentes.

\vspace{0.3cm}
\hrule
\vspace{0.3cm}

% Montgomery, pg 36, cap 2
\item
  \begin{inparaenum}
  \item $10^{-6}$
  \item $0,002$
  \end{inparaenum}

\vspace{0.3cm}
\hrule
\vspace{0.3cm}

% Montgomery, pg 44, cap 2
\item
  \begin{inparaenum}
  \item $0,087$
  \item $0,032$
  \item $0,2$
  \item $0,113$
  \item $0,2207$
  \item $0,005$
  \end{inparaenum}

\vspace{0.3cm}
\hrule
\vspace{0.3cm}


\end{compactenum}

\end{document}
